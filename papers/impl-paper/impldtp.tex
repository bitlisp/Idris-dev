\documentclass{jfp1}
%\documentclass[acmtoplas]{acmtrans2m}

\usepackage[draft]{comments}
%\usepackage[final]{comments}
% \newcommand{\comment}[2]{[#1: #2]}
\newcommand{\khcomment}[1]{\comment{KH}{#1}}
\newcommand{\ebcomment}[1]{\comment{EB}{#1}}

\usepackage{epsfig}
%\usepackage{path}
\usepackage{url}
%\usepackage{amsmath,amssymb} 
\usepackage{fancyvrb}
\usepackage{todonotes}

\newenvironment{template}{\sffamily}

\usepackage{graphics,epsfig}
\usepackage{stmaryrd}

\input{./macros.ltx}
\input{./library.ltx}

\NatPackage
\FinPackage

\newcounter{per}
\setcounter{per}{1}

\newcommand{\Ivor}{\textsc{Ivor}}
\newcommand{\Idris}{\textsc{Idris}}
\newcommand{\IdrisM}{\textsc{Idris}$^-$}
\newcommand{\TT}{\textsf{TT}}
\newcommand{\TTdev}{\textsf{TT$_{dev}$}}
\newcommand{\Funl}{\textsc{Funl}}
\newcommand{\Agda}{\textsc{Agda}}
\newcommand{\LamPi}{$\lambda_\Pi$}

\newcommand{\perule}[1]{\vspace*{0.1cm}\noindent
\begin{center}
\fbox{
\begin{minipage}{7.5cm}\textbf{Rule \theper:} #1\addtocounter{per}{1}
\end{minipage}}
\end{center}
\vspace*{0.1cm}
}

\newcommand{\mysubsubsection}[1]{
\noindent
\textbf{#1}
}
\newcommand{\hdecl}[1]{\texttt{#1}}


\title
  {Implementing General Purpose Dependently Typed Programming Languages}
%\subtitle{Implementing Domain Specific Languages by Overloading}

\author[Edwin Brady]
{EDWIN BRADY\\
School of Computer Science, University of St Andrews, St Andrews,
KY16 9SX, UK}

\begin{document}

\maketitle

\begin{abstract}
Many components of a dependently typed programming language are by now well
understood, for example the underlying type theory, type checking, unification and
evaluation.  How to combine these components into a realistic and usable high
level language is, however, folklore, discovered anew by successive
language implementations.  In this paper, I describe the implementation of a
new dependently typed functional programming language, \Idris{}.
\Idris{} is intended to be a \emph{general purpose} programming language
and as such provides high level concepts such as implicit syntax, 
type classes and \texttt{do} notation. 
I describe the high level language and the underlying type theory, and present
a method for \emph{elaborating} concrete high level syntax with implicit
arguments and type classes into a fully explicit type theory. Furthermore,
I show how this method,
based on a domain specific language embedded in Haskell, facilitates the
implementation of new high level language constructs.

%I describe the implementation of a dependently typed functional
%programming language, \Idris{}. Much has been written about various
%aspects of dependently typed language implementation (e.g. checking
%dependent types, unification, optimisation) but nothing yet about how
%to bring it all together into a complete, practical, usable tool. This paper
%attempts to do so. In particular, I explain what is needed to turn 
%concrete syntax with implicit arguments into fully elaborated type
%theory, using unification and a tactic engine.
\end{abstract}


%\category{D.3.2}{Programming Languages}{Language
%  Classifications}[Applicative (functional) Languages]
%\category{D.3.4}{Programming Languages}{Processors}[Compilers]
%\terms{Languages, Verification, Performance}
%\keywords{Dependent Types, Typechecking}


%\begin{bottomstuff}
%Author's address: Edwin Brady, School of Computer Science, North Haugh, St Andrews,
%KY16 9SX
%\end{bottomstuff}

\section{Introduction}

In conventional programming languages, there is a clear distinction between
\remph{types} and \remph{values}. For example, in Haskell~\cite{haskell-report},
the following are types, representing integers, characters, lists of characters,
and lists of any value respectively:

\begin{itemize}
\item \texttt{Int}, \texttt{Char}, \texttt{[Char]}, \texttt{[a]}
\end{itemize}

\noindent
Correspondingly, the following values are examples of inhabitants of those types:

\begin{itemize}
\item \texttt{42}, \texttt{'a'}, \texttt{"Hello world!"}, \texttt{[2,3,4,5,6]}
\end{itemize}

\noindent
In a language with \emph{dependent types}, however, the distinction is less
clear.  
Dependent types allow types to ``depend'' on values --- in other words,
types are a \emph{first class} language construct and can be manipulated like
any other value. The standard
example is the type of lists of a given length\footnote{Typically, and perhaps
confusingly, referred to in the dependently typed programming literature as
``vectors''}, \texttt{Vect a n}, where \texttt{a} is the element type and
\texttt{n} is the length of the list and can be an arbitrary
term.

When types can contain values, and where those values describe properties (e.g.
the length of a list)
the type of a function can begin to describe its own properties. For example,
concatenating two lists has the property that the resulting list's length is
the sum of the lengths of the two input lists. We can therefore give the following type 
to the  \texttt{app} function, which concatenates vectors: 

\begin{SaveVerbatim}{appendv}

app : Vect a n -> Vect a m -> Vect a (n + m)

\end{SaveVerbatim}
\useverb{appendv}

\noindent
This tutorial introduces \Idris{}, a general purpose functional 
programming language with dependent types.
The goal of the \Idris{} project is to build a dependently typed language suitable
for verifiable \emph{systems} programming. To this end, \Idris{} is a compiled language
which aims to generate efficient executable code. It also has a lightweight foreign
function interface which allows easy interaction with external C libraries.

\subsection{Intended Audience}

This tutorial is intended as a brief introduction to the language, and is aimed
at readers already familiar with a functional language such as Haskell or
OCaml. In particular, a certain amount of familiarity with Haskell syntax is
assumed, although most concepts will at least be explained briefly.  The reader
is also assumed to have some interest in using dependent types for writing and
verifying systems software.

\subsection{Example Code}

This tutorial includes some example code, which has been tested with \Idris{} version
0.9.7. The files are available in the \Idris{} distribution,
so that you can try them out easily,
under \texttt{tutorial/examples}. However, it is strongly recommended that you
type them in yourself, rather than simply loading and reading them.


%\subsection{Overview}

High level view of steps we've taken to implement this new version:

\begin{itemize}
\item Syntax tree for raw and well-typed terms
\item An evaluator for well-typed terms
\item The important bit: a simple type checker. No unification or
  inference.
\begin{itemize}
  \item It is extremely tempting to add implicit arguments and unification to the
        type checker. This is what I did in the first Idris prototype: I learned
        that it was a bad idea! Scope is a big problem.
\end{itemize}
\item The proof state and a tactic engine (Oleg style~\cite{McBride1999}).
\begin{itemize}
  \item Dealing with names: typechecking and evaluation in a context, managing de Bruijn
        indices becomes tricky. Instead: global names, local names during construction,
        and de Bruijn indices when done.
\end{itemize}
\item Unification, and incorporation into the tactic engine.
\begin{itemize}
  \item Note that this can introduce constraints on hole ordering. If we can't satisfy them,
        report an error.
\end{itemize}
\item An Elaborator, as an EDSL, which is a language for applying primitive tactics and
      constructing derived tactics, from a high level syntax withe implicit arguments.
\item Adding primitive operators and a link to Epic~\cite{brady2011epic}.
\item Advanced elaborations: using declarations, where clauses, type classes.
\end{itemize}

For the high level language: we need nothing more than the core type theory,
and a way of putting stuff together. So we typecheck each pattern match clause
individually, in an appropriate context. There is no need for the core type
theory to have pattern matching.


\section{\Idris{} --- the High Level Language}

\label{sect:hll}

\Idris{} is
a pure functional programming language with dependent types. It is
eagerly evaluated by default, and compiled via the Epic supercombinator
library~\cite{brady2011epic}, with irrelevant values and proof terms
automatically erased~\cite{Brady2003,Brady2005}.
In this section, I will give a brief introduction to programming in \Idris{},
covering the most important features. A full tutorial is available elsewhere
\cite{idristutorial}. 

\subsection{Preliminaries}

\Idris{} defines several primitive types: fixed width integers
\tTC{Int}, arbitrary width integers \tTC{Integer}, and
\tTC{Float} for numeric operations, \tTC{Char} and \tTC{String} for
text manipulation, and \tTC{Ptr} which represents foreign pointers.
There are also several data types declared in the library, including
\tTC{Bool}, with values \tDC{True} and \tDC{False}. All of the usual
arithmetic and comparison operators are defined for the primitive types,
and are overloaded using type classes.

An \Idris{} program consists of a module declaration, followed by an optional
list of imports and a collection of definitions and declarations, for example:

\begin{SaveVerbatim}{constprims}

module Main

x : Int
x = 42

main : IO ()
main = putStrLn ("The answer is " ++ show x)

\end{SaveVerbatim}
\useverb{constprims}

\noindent
Like Haskell, the main function is called \texttt{main}, and input and output
is managed with an \texttt{IO} monad. Unlike Haskell, however, \remph{all} top
level functions must have a type signature. This is due to type inference
being, in general, undecidable for languages with dependent types.

A module declaration also opens a \remph{namespace}. The fully qualified names
declared in this module are \texttt{Main.x} and \texttt{Main.main}.


\subsection{Types and Functions}

Data types are declared in a similar way to Haskell data types, with a similar
syntax. Natural numbers and lists, for example, are declared as follows in the
library:

\begin{SaveVerbatim}{natlist}

data Nat    = O   | S Nat           -- Natural numbers
                                    -- (zero and successor)
data List a = Nil | (::) a (List a) -- Polymorphic lists

\end{SaveVerbatim}
\useverb{natlist}

\noindent
Unary natural numbers can be either zero, or
the successor of another natural number (\texttt{S k}). 
Lists can either be empty (\texttt{Nil})
or a value added to the front of another list (\texttt{x :: xs}).
In the declaration for \tTC{List}, we used an infix operator \tDC{::}. New operators
such as this can be added using a fixity declaration, as follows:

\begin{SaveVerbatim}{infixcons}

infixr 10 :: 

\end{SaveVerbatim}
\useverb{infixcons}

\noindent
This declares that \texttt{::} is a right associative operator (\texttt{infixr})
with a precedence level of 10.
Functions, data constructors and type constructors may all be given infix
operators as names. They may be used in prefix form if enclosed in brackets,
e.g. \tDC{(::)}. 

\subsection{Functions}

Functions are implemented by pattern matching, again using a similar syntax to
Haskell. Some natural number arithmetic functions can be
defined as follows, again taken from the standard library:

\begin{SaveVerbatim}{natfns}

-- Unary addition
plus : Nat -> Nat -> Nat
plus O     y = y
plus (S k) y = S (plus k y)

-- Unary multiplication
mult : Nat -> Nat -> Nat
mult O     y = O
mult (S k) y = plus y (mult k y)

\end{SaveVerbatim}
\useverb{natfns}

\noindent
The standard arithmetic operators \texttt{+} and \texttt{*} are also overloaded
for use by \texttt{Nat}, and are implemented
using the above functions.  Unlike Haskell, there is no restriction on whether
types and function names must begin with a capital letter or not. 
%Function
%names (\tFN{plus} and \tFN{mult} above), data constructors (\tDC{O}, \tDC{S},
%\tDC{Nil} and \tDC{::}) and type constructors (\tTC{Nat} and \tTC{List}) are
%all part of the same namespace.

\Idris{} has an interactive prompt, at which we can test these functions:

\begin{SaveVerbatim}{fntest}

Idris> plus (S (S O)) (S (S O))
S (S (S (S O))) : Nat
Idris> mult (S (S (S O))) (plus (S (S O)) (S (S O)))
S (S (S (S (S (S (S (S (S (S (S (S O))))))))))) : Nat

\end{SaveVerbatim}
\useverb{fntest}

\noindent
Like arithmetic operations, integer literals are also overloaded using type classes, 
meaning that we can also test the functions as follows:

\begin{SaveVerbatim}{fntest}

Idris> plus 2 2 
S (S (S (S O))) : Nat
Idris> mult 3 (plus 2 2)
S (S (S (S (S (S (S (S (S (S (S (S O))))))))))) : Nat

\end{SaveVerbatim}
\useverb{fntest}

\subsubsection{\texttt{where} clauses}

Functions can also be defined \emph{locally} using \texttt{where} clauses. For example,
to define a function which reverses a list, we can use an auxiliary function which
accumulates the new, reversed list, and which does not need to be visible globally:

\begin{SaveVerbatim}{revwhere}

reverse : List a -> List a
reverse xs = revAcc [] xs where
  revAcc : List a -> List a -> List a
  revAcc acc [] = acc
  revAcc acc (x :: xs) = revAcc (x :: acc) xs

\end{SaveVerbatim}
\useverb{revwhere}

\noindent
Indentation is significant --- functions in the \texttt{where} block must be indented
further than the outer function.

\textbf{Remark (scope):} 
Any names which are visible in the outer scope are also visible in the
\texttt{where} clause (unless they are redefined in the \texttt{where} clause,
such as \texttt{xs} here). A name which appears only in the type will be in
scope in the \texttt{where} clause if it is a \remph{parameter} to one of the
types, i.e. it is fixed across the entire data structure.  In particular, this means
that the \texttt{a} in the definition of \texttt{reverse} above is the
\remph{same} \texttt{a} as in the definition of \texttt{revAcc}, as \texttt{a}
is a parameter of \texttt{List}.

\subsubsection{Dependent Types}

A standard example of a dependent type is the type of ``lists with length'',
conventionally called ``vectors'' in the dependently typed programming
literature. In \Idris{}, vectors are declared as follows:

\begin{SaveVerbatim}{vect}

data Vect : Type -> Nat -> Type where
   Nil  : Vect a O
   (::) : a -> Vect a k -> Vect a (S k)

\end{SaveVerbatim}
\useverb{vect}

\noindent
Note that this uses the same constructor names as for \tTC{List}. Ad-hoc name
overloading such as this is accepted by \Idris{}, provided that the names are
declared in different namespaces (in practice, normally in different modules)
so that the names are different internally. Namespace resolution can be made
explicitly (e.g. \texttt{List.Nil} or \texttt{Vect.Nil}) or more commonly
by type.

The above declaration creates a family of types, and requires a different form
of declaration from the simple type declarations above. It resembles a Haskell
GADT declaration: it explicitly states the type
of the type constructor \tTC{Vect} --- it takes a type and a \tTC{Nat} as an
argument, where \tTC{Type} stands for the type of types. We say that \tTC{Vect}
is \emph{parameterised} by a type, and \emph{indexed} over \tTC{Nat}. 
The distinction between parameters and indices is that a parameter is fixed
across an entire data structure, whereas an index may vary.
Each constructor targets a different part of the family of types. \tDC{Nil} can
only be used to construct vectors with zero length, and \tDC{::} to construct
vectors with non-zero length. The type of \tDC{::} states explicitly that an element
of type \texttt{a} and a tail of type \texttt{Vect a k} (i.e., a vector of length \texttt{k})
combine to make a vector of length \texttt{S k}.

Functions on dependent types such as \tTC{Vect} are declared in the same way
as on simple types such as \tTC{List} and \tTC{Nat} above, by pattern matching.
The type of a function over \tTC{Vect} will describe what happens to the
lengths of the vectors involved. For example, \tFN{++}, defined in the
library, appends two \tTC{Vect}s:

\begin{SaveVerbatim}{vapp}

(++) : Vect A n -> Vect A m -> Vect A (n + m)
Nil       ++ ys = ys
(x :: xs) ++ ys = x :: xs ++ ys

\end{SaveVerbatim}
\useverb{vapp}

\subsubsection*{Example: The Finite Sets}

Finite sets, as the name suggests, are sets with a finite number of elements.
They are declared as follows in the library:

\begin{SaveVerbatim}{findecl}

data Fin : Nat -> Type where
   fO : Fin (S k)
   fS : Fin k -> Fin (S k)

\end{SaveVerbatim}
\useverb{findecl}

\noindent
This declares
\tDC{fO} as the zeroth element of a finite set with \texttt{S k} elements; 
\texttt{fS n} as the
\texttt{n+1}th element of a finite set with \texttt{S k} elements. 
\tTC{Fin} is indexed by \tTC{Nat}, which
represents the number of elements in the set. 
Neither constructor targets \texttt{Fin O}, because we cannot construct an
element of an empty set.

A useful application of the \tTC{Fin} family is to represent bounded
natural numbers. Since the first \tTC{n} natural numbers form a finite
set of \tTC{n} elements, we can treat \tTC{Fin n} as the set of natural
numbers bounded by \tTC{n}. 

For example, the following function which looks up an element in a \tTC{Vect},
by a bounded index given as a \tTC{Fin n}, is defined in the library:

\begin{SaveVerbatim}{vindex}

index : Fin n -> Vect a n -> a
index fO     (x :: xs) = x
index (fS k) (x :: xs) = index k xs

\end{SaveVerbatim}
\useverb{vindex}

\noindent
This function looks up a value at a given location in a vector. The location is
bounded by the length of the vector (\texttt{n} in each case), so there is no
need for a run-time bounds check. The type checker guarantees that the location
is no larger than the length of the vector.

Note also that there is no case for \texttt{Nil} here. It would be impossible
to apply such a case --- since there is no element of \texttt{Fin O}, and the
location is a \texttt{Fin n}, then \texttt{n} can not be \tDC{O}.  As a result,
attempting to look up an element in an empty vector would give a compile time
type error.

\subsubsection{Implicit Arguments}

Let us take a closer look at the type of \texttt{index}:

\begin{SaveVerbatim}{vindexty}

index : Fin n -> Vect a n -> a

\end{SaveVerbatim}
\useverb{vindexty}

\noindent
It takes two arguments, an element of the finite set of \texttt{n} elements, and a vector
with \texttt{n} elements of type \texttt{a}. But there are also two names, 
\texttt{n} and \texttt{a}, which are not declared explicitly. These are \emph{implicit}
arguments to \texttt{index}. The type of \texttt{index} could also be written as:

\begin{SaveVerbatim}{vindeximppl}

index : {a:_} -> {n:_} -> Fin n -> Vect a n -> a

\end{SaveVerbatim}
\useverb{vindeximppl}

\noindent
This gives bindings for \texttt{a} and \texttt{n} with placeholders for
their types, to be inferred by the machine. These types could also be given explicitly:

\begin{SaveVerbatim}{vindeximpty}

index : {a:Type} -> {n:Nat} -> Fin n -> Vect a n -> a

\end{SaveVerbatim}
\useverb{vindeximpty}

\noindent
Implicit arguments, given in braces \texttt{\{\}} in the type signature, are
not given in applications of \texttt{index}; their values can be inferred from
the types of the \texttt{Fin n} and \texttt{Vect a n} arguments. Any name which
appears as a parameter or index in a type signature, but which is otherwise
free, will be automatically bound as an implicit argument.  Implicit arguments
can still be given explicitly in applications, using the syntax
\texttt{\{a=value\}} and \texttt{\{n=value\}}, for example:

\begin{SaveVerbatim}{vindexexp}

index {a=Int} {n=2} fO (2 :: 3 :: Nil)

\end{SaveVerbatim}
\useverb{vindexexp}

\noindent
In fact, any argument, implicit or explicit, may be given a name. For example,
the type of \texttt{index} could be declared as:

\begin{SaveVerbatim}{vindexn}

index : (i:Fin n) -> (xs:Vect a n) -> a

\end{SaveVerbatim}
\useverb{vindexn}

\noindent
This can be useful for improving the readability of type signatures, particularly
where the name suggests the argument's purpose.

\subsubsection{Totality Checking}

Internally, \Idris{} programs are checked for \emph{totality} --- that they
produce an answer in finite time for all possible inputs --- but they are not
\emph{required} to be total by default. Totality checking serves two
purposes: firstly, if a program terminates for all inputs 
then its type gives a strong
guarantee about the properties specified by its type; secondly, we can
optimise total programs more aggressively~\cite{Brady2003}. 

Totality checking can be enforced by using the \texttt{total} keyword. For
example, recall the \texttt{vAdd} function:

\begin{SaveVerbatim}{vadd}

total vAdd : Num a => Vect a n -> Vect a n -> Vect a n
vAdd []        []        = []
vAdd (x :: xs) (y :: ys) = x + y :: vAdd xs ys

\end{SaveVerbatim}
\useverb{vadd}

\noindent
The elaborator can verify that this is total by checking that it covers all
possible patterns --- in this case, both arguments must be of the same form
as the type requires that the input vectors are the same length --- and that
recursive calls are on structurally smaller values. The totality checker is
implemented independently of the type checker and elaborator presented in the
remainder of this paper. While we could consider building totality proofs
by translating functions to eliminators~\cite{McBride2004a} we have taken
the more pragmatic approach of allowing more flexibility for the programmer
at the expense of simplicity of totality checking. \Idris{} also supports
coinductive definitions with a corresponding productivity checker, although
further details are beyond the scope of this paper.

\subsection{Type Classes}

\Idris{} supports overloading in two ways. Firstly, as we have already seen with
the constructors of \texttt{List} and \texttt{Vect}, names
can be overloaded in an ad-hoc manner and resolved according to the context in which
they are used. This is mostly for convenience, to eliminate the need to decorate
constructor names in similarly structured data types, and eliminate explicit qualification
of ambiguous names where only one is well-typed --- this is especially useful
for disambiguating record field names\footnote{Records are however beyond the scope
of this paper}.

Secondly, \Idris{} implements \remph{type classes}, following Haskell.  This
allows a more principled approach to overloading --- a type class gives a
collection of overloaded operations which describe the interface for
\remph{instances} of that class.

A simple example
is the \texttt{Show} type class, which is defined in the library and
provides an interface for converting values to
\texttt{String}s:

\begin{SaveVerbatim}{showclass}

class Show a where
    show : a -> String

\end{SaveVerbatim}
\useverb{showclass}

\noindent
This declares a function of the following type (which we call a \emph{method} of the 
\texttt{Show} class):

\begin{SaveVerbatim}{showty}

show : Show a => a -> String

\end{SaveVerbatim}
\useverb{showty}

An instance of a class
is defined with an \texttt{instance} declaration, which provides implementations of
the function for a specific type. For example, the \texttt{Show} instance for \texttt{Nat}
could be defined as:

\begin{SaveVerbatim}{shownat}

instance Show Nat where
    show O = "O"
    show (S k) = "s" ++ show k

\end{SaveVerbatim}
\useverb{shownat}

\begin{SaveVerbatim}{shownati}

Idris> show (S (S (S O))) 
"sssO" : String

\end{SaveVerbatim}
\useverb{shownati}

\noindent
Only one instance of a class can be given for a type --- instances may not overlap.
Instance declarations can themselves have constraints. For example, to define a
\texttt{Show} instance for vectors, we need to know that there is a \texttt{Show} 
instance for the element type, because we are going to use it to convert each element
to a \texttt{String}:

\begin{SaveVerbatim}{showvec}

instance Show a => Show (Vect a n) where
    show xs = "[" ++ show' xs ++ "]" where
        show' : Vect a n -> String
        show' Nil        = ""
        show' (x :: Nil) = show x
        show' (x :: xs)  = show x ++ ", " ++ show' xs

\end{SaveVerbatim}
\useverb{showvec}

%\noindent
%\textbf{Remark: } The type of the auxiliary function \texttt{show'} is
%important. The type variables \texttt{a} and \texttt{n} which are part of the
%instance declaration for \texttt{Show (Vect a n)} are fixed across the entire
%instance declaration. As a result, \texttt{a} need not be constrained
%again. Furthermore, it means that if \texttt{n} is used again in the type, it refers
%to the (fixed) length of the outermost list \texttt{xs}. Therefore,
%there is a different name for the length \texttt{n'} in \texttt{show'}.

\noindent
Like Haskell type classes, default definitions can be given in the class declaration.
Otherwise, all methods must be given in an instance. For example, there is an
\texttt{Eq} class:

\begin{SaveVerbatim}{eqdefault}

class Eq a where
    (==) : a -> a -> Bool
    (/=) : a -> a -> Bool

    x /= y = not (x == y)
    y == y = not (x /= y)

\end{SaveVerbatim}
\useverb{eqdefault}

\noindent
Classes can also be extended. A logical next step from an equality relation \texttt{Eq}
is to define an ordering relation \texttt{Ord}. We can define an \texttt{Ord} class
which inherits methods from \texttt{Eq} as well as defining some of its own:

\begin{SaveVerbatim}{ord}

data Ordering = LT | EQ | GT

\end{SaveVerbatim}
\useverb{ord} 

\begin{SaveVerbatim}{eqord}

class Eq a => Ord a where
    compare : a -> a -> Ordering
    (<) : a -> a -> Bool
    -- etc

\end{SaveVerbatim}
\useverb{eqord}

\subsection{Matching on intermediate values}

%\subsubsection{\texttt{let} bindings}
%
%Intermediate values can be calculated using \texttt{let} bindings:
%
%\begin{SaveVerbatim}{letb}
%
%mirror : List a -> List a
%mirror xs = let xs' = rev xs in
%                app xs xs'
%
%\end{SaveVerbatim}
%\useverb{letb} 
%
%\noindent
%Pattern matching is also supported in \texttt{let} bindings. For example, extracting
%fields from a record can be achieved as follows, as well as by pattern matching at the top level:
%
%\begin{SaveVerbatim}{letp}
%
%data Person = MkPerson String Int
%
%showPerson : Person -> String
%showPerson p = let MkPerson name age = p in
%                   name ++ " is " ++ show age ++ " years old"
%
%\end{SaveVerbatim}
%\useverb{letp} 

\subsubsection{\texttt{case} expressions}

Intermediate values of \emph{non-dependent} types can be inspected using a
\texttt{case} expression.  For example, \texttt{list\_lookup} looks up an index
in a list, returning \texttt{Nothing} if the index is out of bounds. This can
be used to write \texttt{lookup\_default}, which looks up an index and
returns a default value if the index is out of bounds:

\begin{SaveVerbatim}{listlookup}

lookup_default : Nat -> List a -> a -> a
lookup_default i xs def = case list_lookup i xs of
                              Nothing => def
                              Just x => x

\end{SaveVerbatim}
\useverb{listlookup} 

The \texttt{case} construct is intended for simple analysis of intermediate
expressions to avoid the need to write auxiliary functions.  It will
\emph{only} work if each branch \emph{matches} a value of the same type, and
\emph{returns} a value of the same type.

\subsubsection{The \texttt{with} rule}

Since types can depend on values, the form of some arguments can be determined
by the value of others. For example, if we were to write down the implicit
length arguments to \texttt{(++)}, we would see that the form of the length argument was
determined by whether the vector was empty or not:

\begin{SaveVerbatim}{appdep}

(++) : Vect a n -> Vect a m -> Vect a (n + m)
(++) {n=O}   []        [] = []
(++) {n=S k} (x :: xs) ys = x :: xs ++ ys

\end{SaveVerbatim}
\useverb{appdep}

\noindent
If \texttt{n} was a successor in the \texttt{[]} case, or zero in the 
\texttt{::} case, the definition
would not be well typed.

Often, matching is required on the result of an intermediate computation
with a dependent type.
\Idris{} provides a construct for this, the \texttt{with} rule, 
inspired by views in \Epigram~\cite{McBride2004a},
which takes account of the
fact that matching on a value in a dependently typed language can affect what
is known about the forms of other values. 

For example, a \texttt{Nat} is either even or odd. 
If it is even it will
be the sum of two equal \texttt{Nat}s. Otherwise, it is the sum of two equal \texttt{Nat}s 
plus one:

\begin{SaveVerbatim}{parity}

data Parity : Nat -> Type where
   even : Parity (n + n)
   odd  : Parity (S (n + n))

\end{SaveVerbatim}
\useverb{parity}

\noindent
We say \texttt{Parity} is a \emph{view} of \texttt{Nat}. 
It has a \emph{covering function} which tests whether
it is even or odd and constructs the predicate accordingly.

\begin{SaveVerbatim}{parityty}

parity : (n:Nat) -> Parity n

\end{SaveVerbatim}
\useverb{parityty}

\noindent
Using this, a function which converts a natural number to a list
of binary digits (least significant first) is written as follows, using the \texttt{with}
rule:

\begin{SaveVerbatim}{natToBin}

natToBin : Nat -> List Bool
natToBin O = Nil
natToBin k with (parity k)
   natToBin (j + j)     | even = False :: natToBin j
   natToBin (S (j + j)) | odd  = True  :: natToBin j

\end{SaveVerbatim}
\useverb{natToBin}

\noindent
The value of the result of \texttt{parity k} affects the form of \texttt{k}, 
because the result
of \texttt{parity k} depends on \texttt{k}. 
So, as well as the patterns for the result of the
intermediate computation (\texttt{even} and \texttt{odd}) right of the 
\texttt{$\mid$}, the definition also expresses how
the results affect the other patterns left of the $\mid$. Note that there is a
function in the patterns (\texttt{+}) and repeated occurrences of \texttt{j} --- 
this is allowed
because another argument has determined the form of these patterns.




\section{The Core Type Theory}

\label{sect:typechecking}

High level \Idris{} programs, as described in Section \ref{sect:hll}, are 
\remph{elaborated} to a small core language, \TT{}, then type checked. 
\TT{} is a dependently typed $\lambda$-calculus with inductive families
and pattern matching definitions similar to UTT~\cite{luo1994}, and building
on an earlier implementation, \Ivor{}~\cite{Brady2006b}.
The
core language is kept as small as is reasonably possible, which has several advantages: it is
easy to type check, since type checking dependent type theory is well understood
~\cite{loh2010tutorial}; and it is easy to transform, optimise and compile. Keeping
the core language small increases confidence that these important components of
the language are correct. In this section, we describe \TT{} and
its semantics.

\subsection{\TT{} syntax}

The syntax of \TT{} expressions is given in Figure \ref{ttsyn}. This defines:

\begin{itemize}
\item \demph{Terms}, $\vt$, which can be variables, bindings, applications or constants.
\item \demph{Bindings}, $\vb$, which can be lambda abstractions, let bindings, or function spaces.
\item \demph{Constants}, $\vc$, which can be integer or string literals, or $\Type_i$, the
type of types, and may be extended with other primitives. 
\end{itemize}

The function space $\all{\vx}{\vS}\SC\vT$ may also be written as $(\vx\Hab\vS)\to\vT$,
or $\vS\to\vT$ if $\vx$ is not free in $\vT$, to make the notation more consistent with
the high level language. Universe levels on types ($\Type_i$) may be left implicit and
inferred by the machine~\cite{pollack1990implicit}.

\FFIG{
\begin{array}{rll@{\hg}rll}
\mbox{Terms,}\;\vt ::= & \vc & (\mbox{constant}) &
\;\mbox{Binders,}\;\vb ::= & \lam{\vx}{\vt} & (\mbox{abstraction}) \\

 \mid  & \vx & (\mbox{variable}) &
 \mid & \LET\;\vx\defq\vt\Hab\vt & (\mbox{let binding}) \\

 \mid   & \vb\SC\;\vt & (\mbox{binding}) &
 \mid & \all{\vx}{\vt} & (\mbox{function space}) \\

 \mid   & \vt\;\vt & (\mbox{application}) &
% \mid & \pat{\vx}{\vt} & (\mbox{pattern variable}) \\
% & & &
% \mid & \pty{\vx}{\vt} & (\mbox{pattern type}) \\
\medskip\\
\mbox{Constants,}\;\vc ::= & \Type_i & (\mbox{type universes}) \\
   \mid & \vi & (\mbox{integer literal}) \\
   \mid & \VV{str} & (\mbox{string literal}) \\
\end{array}
}
{\TT{} expression syntax}
{ttsyn}

A \TT{} program is a collection of \demph{inductive family} definitions (Section 
\ref{sect:inductivefams}) and \demph{pattern matching} function definitions (Section
\ref{sect:patdefs}), as well as primitive operators on constants. 
Before defining these, let us define the semantics of \TT{}
expressions.

\subsection{\TT{} semantics}

The static and dynamic semantics of \TT{} are defined mutually, since
evaluation relies on a well-typed term, and type checking relies on 
evaluation. Everything is defined relative
to a context, $\Gamma$. The empty context
is valid, as is a context extended with a $\lambda$, $\forall$ or
$\LET$ binding:

\DM{
\Axiom{\proves\RW{valid}}
\hg
\Rule{\Gamma\proves\vS\Hab\Type_i}
{\Gamma;\lam{\vx}{\vS}\proves\RW{valid}}
\hg
\Rule{\Gamma\proves\vS\Hab\Type_i}
{\Gamma;\all{\vx}{\vS}\proves\RW{valid}}
\hg
\Rule{\Gamma\proves\vS\Hab\Type_i\hg\Gamma\proves\vs\Hab\vS}
{\Gamma;\LET\;\vx\;\defq\;\vs\Hab\vS\proves\RW{valid}}
}

\subsubsection{Evaluation}

\label{sect:evaluation}

Evaluation of \TT{} is defined by contraction schemes, given in Figure
\ref{ttcontract}. \demph{Contraction}, relative to a context $\Gamma$, is given
by one of the following schemes:

\begin{itemize}
\item $\beta$-contraction, which substitutes a value applied to a $\lambda$-binding for
the bound variable. 
%We define $\beta$-contraction simply by replacing the $\lambda$-binding
%with a $\LET$ binding.
\item $\delta$-contraction, which replaces a $\LET$ bound variable with its value.
\end{itemize}

\noindent
The following contextual closure rule reduces a $\LET$ binding by creating
a $\delta$-reducible expression:

\DM{
\Rule{\Gamma;\LET\;\vx\defq\vs\Hab\vS\proves\vt\leadsto\vu}
{\Gamma\proves\LET\;\vx\defq\vs\Hab\vS\SC\vt\leadsto\vu}
}

\demph{Reduction} ($\reduces$) is the structural closure of contraction, and evaluation
is ($\reducesto$) is the transitive closure of reduction. \demph{Conversion} ($\converts$)
is the smallest equivalence relation closed under reduction. If $\Gamma\proves\vx\converts\vy$
then $\vy$ can be obtained from $\vx$ by a finite, possibly empty, sequence of
contractions and reversed contractions. Terms which are convertible are also said to
be definitionally equal.
The evaluator can also be extended by defining pattern matching functions, which
will be described in more detail in Section \ref{sect:patdefs}. In principle, pattern
matching functions can be understood as extending the core language with high level
reduction rules.

\FFIG{
\begin{array}{lc}
\beta\mathrm{-contraction} &
\Axiom{
\Gamma\proves(\lam{\vx}{\vS}\SC\vt)\;\vs\leadsto
\vt[\vs/\vx]
%\LET\;\vx\defq\vs\Hab\vS\SC\vt
} \\
\delta\mathrm{-contraction} &
\Axiom{
\Gamma;\LET\;\vx\defq\vs\Hab\vS;\Gamma'
\proves
\vx\leadsto\vs
}
\end{array}
}
{\TT{} contraction schemes}
{ttcontract}



\subsubsection{Typing rules}

\label{sect:typerules}

The type inference rules for \TT{} expressions are given in Figure
\ref{typerules}.  These rules use the \demph{cumulativity} relation, defined in
Figure \ref{cumul}. The type of types, $\Type_i$ is parameterised by a universe
level, to prevent Girard's paradox~\cite{coquand1986analysis}.  There is an
infinite hierarchy of predicative universes.  Cumulativity allows programs at
lower universe levels to inhabit higher universe levels. In practice, universe levels
can be left implicit (and will be left implicit in the rest of this paper) ---
the type checker generates a graph of constraints between universe levels (such
as that produced by the $\mathsf{Forall}$ typing rule) and checks that there
are no cycles. Otherwise, the typing rules are standard and type inference can
be implemented in the usual way~\cite{loh2010tutorial}.

\FFIG{\begin{array}{c}
\mathsf{Type}\;
\Rule{\Gamma\proves\RW{valid}}
{\Gamma\vdash\Type_n\Hab\Type_{n+1}}
\\
\mathsf{Const}_1\;
\Rule{\Gamma\proves\RW{valid}}
{\Gamma\vdash\vi\Hab\TC{Int}}
\hg
\mathsf{Const}_2\;
\Rule{\Gamma\proves\RW{valid}}
{\Gamma\vdash\VV{str}\Hab\TC{String}}
\\
\mathsf{Const}_3\;
\Rule{\Gamma\proves\RW{valid}}
{\Gamma\vdash\TC{Int}\Hab\Type_0}
\hg
\mathsf{Const}_4\;
\Rule{\Gamma\proves\RW{valid}}
{\Gamma\vdash\TC{String}\Hab\Type_0}
\\
%\mathsf{Pat}_1\;
%\Rule{(\pat{\vx}{\vS})\in\Gamma}
%{\Gamma\vdash\vx\Hab\vS}
%\hg
%\mathsf{Pat}_2\;
%\Rule{(\pty{\vx}{\vS})\in\Gamma}
%{\Gamma\vdash\vx\Hab\vS}
\\
\mathsf{Var}_1\;
\Rule{(\lam{\vx}{\vS})\in\Gamma}
{\Gamma\vdash\vx\Hab\vS}
\hg
\mathsf{Var}_2\;
\Rule{(\all{\vx}{\vS})\in\Gamma}
{\Gamma\vdash\vx\Hab\vS}
\hg
\mathsf{Val}\;
\Rule{(\LET\;\vx\defq\vs\Hab\vS)\in\Gamma}
{\Gamma\vdash\vx\Hab\vS}
\\
\mathsf{App}\;
\Rule{\Gamma\vdash\vf\Hab\fbind{\vx}{\vS}{\vT}\hg\Gamma\vdash\vs\Hab\vS}
{\Gamma\vdash\vf\;\vs\Hab\vT[\vs/\vx]} % \LET\;\vx\Hab\vS\;\defq\;\vs\;\IN\;\vT}
\\
\mathsf{Lam}\;
\Rule{\Gamma;\lam{\vx}{\vS}\vdash\ve\Hab\vT\hg\Gamma\proves\fbind{\vx}{\vS}{\vT}\Hab\Type_n}
{\Gamma\vdash\lam{\vx}{\vS}.\ve\Hab\fbind{\vx}{\vS}{\vT}}
\\
\mathsf{Forall}\;
\Rule{\Gamma;\all{\vx}{\vS}\vdash\vT\Hab\Type_m\hg\Gamma\vdash\vS\Hab\Type_n}
{\Gamma\vdash\fbind{\vx}{\vS}{\vT}\Hab\Type_p}
\;(\exists\vp.\vm\le\vp,\;\vn\le\vp)
\\
\mathsf{Let}\;
\Rule{\begin{array}{c}\Gamma\proves\ve_1\Hab\vS\hg
      \Gamma;\LET\;\vx\defq\ve_1\Hab\vS\proves\ve_2\Hab\vT\\
      \Gamma\proves\vS\Hab\Type_n\hg
      \Gamma;\LET\;\vx\defq\ve_1\Hab\vS\proves\vT\Hab\Type_n\end{array}
      }
{\Gamma\vdash\LET\;\vx\defq\ve_1\Hab\vS\SC\;\ve_2\Hab
   \vT[\ve_1/\vx]}   
%\Let\;\vx\Hab\vS\defq\ve_1\;\IN\;\vT}
\\

\mathsf{Conv}\;
\Rule{\Gamma\proves\vx\Hab\vA\hg\Gamma\proves\vA'\Hab\Type_n\hg
      \Gamma\proves\vA\cumul\vA'}
     {\Gamma\proves\vx\Hab\vA'}
\end{array}
}
{Typing rules for \TT{}}
{typerules}

\FFIG{
\begin{array}{c}
\Rule{\Gamma\proves\vS\converts\vT}
{\Gamma\proves\vS\cumul\vT}
\hg
\Axiom{\Gamma\proves\Type_n\cumul\Type_{n+1}}
\\
\Rule{\Gamma\proves\vR\cumul\vS\hg\Gamma\proves\vS\cumul\vT}
{\Gamma\proves\vR\cumul\vT}
\\
\Rule{\Gamma\proves\vS_1\converts\vS_2\hg\Gamma;\vx\Hab\vS_1\proves\vT_1\cumul\vT_2}
{\Gamma\proves\all{\vx}{\vS_1}\SC\vT_1\cumul\all{\vx}{\vS_2}{\vT_2}}
\end{array}
}
{Cumulativity}
{cumul}


\subsection{Inductive Families}

\label{sect:inductivefams}

Inductive families \cite{dybjer1994inductive} are a form of simultaneously
defined collection of algebraic data types which can be parameterised over
\remph{values} as well as types.  An inductive family is declared 
in a similar style to a Haskell GADT declaration~\cite{pj2006gadts}
as
follows, using vector notation, $\tx$, to indicate a
sequence of zero or more $\vx$ (i.e., $\vx_1,\vx_2,\ldots,\vx_n$):

\DM{
\AR{
\Data\hg\TC{T}\;(\tx\Hab\ttt)\Hab\vt\hg\Where\hg
\DC{c}_1\Hab\vt\;\mid\;\ldots\;\mid\;\DC{c}_n\Hab\vt
}
}

Constructors may take recursive arguments in the family $\TC{T}$, which may be
higher order. Higher order recursive arguments may be computed from any type
which does not involve $\TC{T}$ itself. This restriction is known as
\demph{strict positivity} and ensures that recursive arguments of the
constructor are structurally smaller than the value itself.

For example, the \Idris{} data type \tTC{Nat} would be declared in \TT{} as follows:

\DM{
\Data\hg\Nat\Hab\Type\hg\Where\hg\Z\Hab\Nat\;\mid\;\suc\Hab\fbind{\vk}{\Nat}{\Nat}
}

A data type may have zero or more parameters (which are invariant
across a structure) and a number of indices, given by the type. For
example, the \TT{} equivalent of \tTC{List} is parameterised over its element type:

\DM{
\AR{
\Data\hg\List\:(\va\Hab\Type)\Hab\Type\hg\Where\\
\hg\hg
\ARd{
& \nil\Hab\List\;\va\\
\mid & (\cons)\Hab\fbind{\vx}{\va}{\fbind{\vxs}{\List\;\va}{\List\;\va}}
}
}
}

Types can be
parameterised over values. Using this, we can declare the type of
vectors (lists with length), where the empty list is statically known
to have length zero, and the non empty list is statically known to
have a non zero length. The \TT{} equivalent of \tTC{Vect} is parameterised over its element type,
like $\List$, but \remph{indexed} over its length. Note also that the length
index $\vk$ is given \remph{explicitly}.

\DM{
\AR{
\Data\hg\Vect\;(\va\Hab\Type)\Hab\Nat\to\Type\hg\Where \\
\hg\hg\ARd{
& \nil\Hab\Vect\;\va\;\Z\\
\mid & (\cons)\Hab\fbind{\vk}{\Nat}{
\fbind{\vx}{\va}{\fbind{\vxs}{\Vect\;\va\;\vk}{\Vect\;\va\;(\suc\;\vk)}}
}
}
}
}

\subsection{Pattern matching definitions}

\label{sect:patdefs}

%\subsubsection{Syntax}


A pattern matching definition for a function named $\FN{f}$ takes the following form,
consisting of a type declaration followed by one or more pattern clauses:

\DM{
\AR{
\FN{f}\Hab\vt\\
\pat{\tx_1}{\ttt_1}\SC\FN{f}\;\ttt_1\;=\;\vt_1\\
\ldots\\
\pat{\tx_n}{\ttt_n}\SC\FN{f}\;\ttt_n\;=\;\vt_n\\
}
}

A pattern clause consists of a list of pattern variable bindings, introduced by
$\RW{var}$,
and a left and right hand side, both of which
are \TT{} expressions. Each side is type checked relative to the variable bindings,
and the types of each side must be convertible. Additionally, the
left hand side must take the form of $\FN{f}$ applied to a number of \TT{} expressions,
and the number of arguments must be the same in each clause. The right hand
sides may include applications of $\FN{f}$, i.e. pattern matching definitions may
be recursive. Termination analysis is implemented separately. The validity of a pattern
clause is defined by the following rule:

\DM{
\Rule{
\Gamma;\lam{\tx}{\tU}\proves\FN{f}\;\tts\Hab\vS\hg
\Gamma;\lam{\tx}{\tU}\proves\ve\Hab\vT\hg
\Gamma\proves\vS\converts\vT}
{
\Gamma\proves\pat{\tx}{\tU}\SC\FN{f}\;\tts\;=\;\ve\;\RW{valid}
}
}

A valid pattern matching definition effectively extends \TT{} with a new
constant, with the given type (extending the initial typing rules given in
Section \ref{sect:typerules}) and reduction behaviour (extending the initial
reduction rules given in Section \ref{sect:evaluation}). 
Patterns are separated into the accessible patterns (variables and constructor
forms which may be inspected) and inaccessible patterns, following
Agda~\cite{norell2007thesis} then implemented by compilation into case
trees~\cite{Augustsson1985}.

% \subsubsection{Semantics}
% 
% Matching on an expression proceeds by comparing the expression to
% each match clause in order, resulting in either:
% 
% \begin{itemize}
% \item \demph{Success}, with pattern variables mapping to expressions
% \item \demph{Failure}, with matching continuing by proceeding to the next match clause
% \item Matching being \demph{blocked}, for example by attempting to match a variable
% against a constructor pattern. In this case, no reduction occurs, because instantiating
% the variable may provide enough information for the clause to match.
% \end{itemize}
% 
% % Pat(x args) => x (Pat' args)
% % Pat (x) => x
% % Pat _ => _
% 
% In order to implement pattern matching, we must separate the \remph{accessible} and
% \remph{inaccessible} patterns. A pattern is \demph{accessible} (that is, it is possible
% to match against it) if it is constructor headed, or a variable. Inaccessible patterns
% are converted to ``match anything'' patterns. Clauses are converted to matchable pattern
% clauses with the $\MO{Clause}$ operation, given in Figure \ref{mkclause}.
% We extend the vector notation to meta-operations: the notation $\vec{\MO{Pat}}$ lifts the
% $\MO{Pat}$ operation across a list of arguments.
% 
% \FFIG{
% \AR{
% \PA{\A}{
% & \MO{Clause} & (\pat{\tx}{\tU}\SC\FN{f}\;\tts\;=\;\ve)
%  & \MoRet{\FN{f}\;(\vec{\MO{Pat}}\;\tts)\;=\;\ve}
% }
% \medskip\\
% \PA{\A}{
% & \MO{Pat} & (\vx\;\tts) & \MoRet{\vx\;(\vec{\MO{Pat}}\;\tts) 
%   \hg\mbox{(if $\MO{Con}\;\vx$ and $\MO{Arity}\;\vx = \MO{Length}\;\tts$)}} \\
% & \MO{Pat} & \vx & \MoRet{\vx} \hg\mbox(if $\vx$ is a pattern variable)\\
% & \MO{Pat} & \cdot & \MoRet{\_} \hg\mbox(in all other cases)\\
% }
% }
% }
% {Building matchable pattern clauses}
% {mkclause}
% 
% 
% % Match (c args, c args') => Match' (args, args')
% % Match (c args, x)       => Blocked
% % Match (x, t)            => Success (x => t)
% % Match (_, t)            => Success ()  
% 
% \newcommand{\Blocked}{\mathsf{Blocked}}
% \newcommand{\Success}{\mathsf{Success}}
% \newcommand{\Failure}{\mathsf{Failure}}
% 
% Pattern matching is invoked by the evaluator on encountering a function with
% a pattern matching definition applied to some arguments. If successful, pattern
% matching returns a new expression, which can then be reduced further.
% Figure \ref{pmatch} gives the semantics of pattern matching a collection of 
% matchable clauses
% $\tc$ against an expression $\ve$. $\MO{Match}$ attempts to match each clause in
% turn against $\ve$. If matching a clause returns $\Success$ or $\Blocked$, then
% matching terminates. If matching a clause returns $\Failure$, then matching proceeds
% with the next clause. The algorithm uses the following meta-operations:
% 
% \begin{itemize}
% \item $\MO{Con}\;\vx$, which returns true if $\vx$ is a constructor name
% \item $\MO{Arity}\;\vx$, which, if $\vx$ is a constructor name, returns the number of arguments
% $\vx$ requires.
% \item $\MO{Length}\;\tts$, which returns the length of the vector $\tts$
% \end{itemize}
% 
% 
% \FFIG{
% \AR{
% \PA{\A\A}{
% & \MO{MatchArg} & (\vx\;\tts) & (\vx\;\tts')
%     & \MoRet{\vec{\MO{MatchArg}}\;\tts\;\tts'\hg\mbox{(if $\MO{Con}\;\vx$)}}\\
% & \MO{MatchArg} & (\vx\;\tts) & \vx & \MoRet{\Blocked} \\
% & \MO{MatchArg} & \vx & \vt & \MoRet{\Success\;(\vx\mapsto\vt)} \\
% & \MO{MatchArg} & \_ & \vt & \MoRet{\Success\;()} \\
% & \MO{MatchArg} & \cdot & \cdot & \MoRet{\Failure} 
% }
% \medskip\\
% \PA{\A\A}{
% & \MO{MatchClause} & (\vf\;\tp = \ve) & (\vf\;\tts) &
%  \MoRet{
%  \AR{
%  \Success\;\ve[\tx/\tv]\hg\mbox{(if $\vec{\MO{MatchArg}}\;\tp\;\tts \mq \Success\;(\tx,\tv)$)}\\
%  \Blocked\;\;\;\hg\hg\mbox{(if $\vec{\MO{MatchArg}}\;\tp\;\tts \mq \Blocked$)}\\
%  \Failure\;\;\;\;\;\hg\hg\mbox{(if $\vec{\MO{MatchArg}}\;\tp\;\tts \mq \Failure$)}\\
%  }
%  } \\
% & \MO{MatchClause} & (\vf\;\tp = \ve) & \cdot & \MoRet{\Failure} \\
% }
% \medskip\\
% \PA{\A\A}{
% & \MO{Match} & (\vc ; \tc) & \ve &
%  \MoRet{
%  \AR{
%  \Success\;\vx\;\;\hg\mbox{(if $\MO{MatchClause}\;\vc\;\ve \mq \Success\;\vx)$} \\
%  \Blocked\hg\;\;\;\;\mbox{(if $\MO{MatchClause}\;\vc\;\ve \mq \Blocked)$} \\
%  \MO{Match}\;\tc\;\ve\hg\mbox{(otherwise)}
%  }
%  } \\
% & \MO{Match} & \cdot & \ve & \MoRet{\Failure}
% }
% }
% }
% {Pattern matching semantics}
% {pmatch}
% 
% By convention, we write the application of a meta-operation $\MO{Op}$ 
% across a vector $\tx$ as $\vec{\MO{Op}}\;\tx$.
% In practice, for efficiency, pattern matching is implemented by compiling the match clauses to
% a tree of case expressions~\cite{Augustsson1985}. 

\subsection{Metatheory}

\FFIG{
\begin{array}{ll}
\mbox{\textbf{Church-Rosser}} &
\begin{array}{ll}
\mbox{If}\hg\Gamma\proves\vs\converts\vt \\
\mbox{then there is a common reduct $\vr$ such that}\;\\
\Gamma\proves\vs\reducesto\vr\hg\Gamma\proves\vt\reducesto\vr
\end{array}
\medskip
\\
\mbox{\textbf{Subject Reduction}} &
\begin{array}{ll}
\mbox{If} & \Gamma\proves\vs\Hab\vS\hg\Gamma\proves\vs\reduces\vt\\
\mbox{then}& \Gamma\proves\vt\Hab\vS
\end{array}
\medskip
\\
\mbox{\textbf{Cut}} &
\begin{array}{ll}
\mbox{If} & \Gamma_0,\LET\;\vx\defq\vs\Hab\vS,\Gamma_1\proves\vt\Hab\vT\\
\mbox{then} & \Gamma_0, \Gamma_1[\vs/\vx]\proves\vt[\vs/\vx]\Hab\vT[\vs/\vx]
\end{array}
\medskip
\\
\mbox{\textbf{Strengthening}} &
\begin{array}{ll}
\mbox{If} & \Gamma_0,\mathcal{B}(\vx, \vT),\Gamma_1\proves\vs\Hab\vS
\hg\vx\not\in\Gamma_1,\vs,\vS\\
\mbox{then} & \Gamma_0,\Gamma_1\proves\vs\Hab\vS\\
(\mbox{where} & \mathcal{B}(\vx, \vT) \proves \vx\Hab\vT)
\end{array}
\\
\end{array}
}
{Metatheoretic properties of \TT{}}
{metatheory}

We conjecture that \TT{} respects the usual metatheoretic properties, as shown in
Figure \ref{metatheory}.  Specifically: the \demph{Church-Rosser} property
(i.e. that distinct reduction sequences lead to the same normal form);
\demph{subject reduction} (i.e. that computation preserves type); the
\demph{cut} property (i.e. that $\RW{let}$-bound terms may be substituted into
their scope); and \demph{strengthening} (i.e. that removing unused definitions
from scope does not affect type checking).

One property which is noticeably absent is \demph{strong normalisation}.
Although desirable to preserve termination and decidability of type checking,
we allow diverging terms for pragmatic reasons as \Idris{} is primarily a
programming language rather than a theorem prover. Nevertheless, we do
\emph{check} for totality in order to be sure which terms are safe to
evaluate at compile time.

\subsection{Totality checking}

In order to ensure termination of type checking we must distinguish terms for
which evaluation definitely terminates, and those which may diverge. \TT{}
takes a simple but pragmatic and effective approach to termination checking:
any functions which do not satisfy a syntactic constraint on recursive calls
are marked as \emph{partial}. Additionally, any function which calls a partial
function or uses a data type which is not strictly positive is also marked as
partial. We use the size change principle~\cite{Lee2001} to determine whether
(possibly mutually defined) recursive functions are guaranteed to
terminate.
\TT{} also marks functions which do not cover
all possible inputs as partial. This totality checking is independent of the
rest of the type theory, and can be extended.

This approach, separating the termination requirement from the type theory,
means that an \Idris{} programmer makes the decision about the importance of
totality for each function rather than having the totality requirement imposed
by the type theory.

\subsection{From \Idris{} to \TT{}}

\TT{} is a very small language, consisting only of data declarations and pattern matching
function definitions. There is no significant innovation in the design of \TT{}, and this
is a deliberate choice --- it is a combination of small, well-understood components.
The kernel of the \TT{} implementation, consisting of a type checker, evaluator and
pattern match compiler, is less than 1000 lines of Haskell code. If we are confident
in the correctness of the kernel of \TT{}, and any higher level language feature
can be translated into \TT{}, we can be more confident of the correctness of the high
level language implementation than if it were implemented directly.

The process of elaborating a high level language into a core type theory like \TT{} is,
however, less well understood, and presents several challenges depending on the
features of the high level language. 
%Elaboration has been implemented in various different ways, 
%for example in Agda~\cite{norell2007thesis} and Epigram~\cite{McBride2004a}.

\subsubsection{Example Elaboration}

Recall the following \Idris{} function:

\useverb{vadd} 

\noindent
In order to elaborate
this to \TT{}, we must resolve the implicit arguments, and make the type class explicit.
The first step is to make the implicit arguments explicit, using a placeholder
to stand for the arguments we have not yet resolved. The type class argument is
also treated as an implicit argument, to which we give the name \texttt{c}:

\begin{SaveVerbatim}{vAddImp}

vAdd : (a : _) -> (n : _) -> Num a -> Vect a n -> Vect a n -> Vect a n
vAdd _ _ c (Nil _)         (Nil _)         = (Nil _)
vAdd _ _ c ((::) _ _ x xs) ((::) _ _ y ys) 
                = (::) _ _ ((+) _ x y) (vAdd _ _ _ xs ys)

\end{SaveVerbatim}
\useverb{vAddImp} 

Next, we resolve the implicit arguments. Each implicit argument can only take
one value for this program to be type correct --- these are solved by a unification
algorithm:

\begin{SaveVerbatim}{vAddImpSolve}

vAdd : (a : Type) -> (n : Nat) -> Num a -> Vect a n -> Vect a n -> Vect a n
vAdd a O     c (Nil a)         (Nil a)         = Nil a
vAdd a (S k) c ((::) a k x xs) ((::) a k y ys) 
                = (::) a k ((+) c x y) (vAdd a k c xs ys)

\end{SaveVerbatim}
\useverb{vAddImpSolve} 

Finally, to build the \TT{} definition, we need to find the type of each
pattern variable and state it explicitly. This leads to the following
\TT{} definition, switching to \TT{} notation from the ASCII \Idris{}
syntax:

\DM{
\AR{
\FN{vAdd} \Hab(\va\Hab\Type)\to(\vn\Hab\Nat)\to\TC{Num}\;\va\to
  \Vect\;\va\;\vn\to\Vect\;\va\;\vn\to\Vect\;\va\;\vn\\
\RW{var}\;\va\Hab\Type,\;\vc\Hab\TC{Num}\;\va\SC\\
\hg\FN{vAdd} \; \va \; \Z \; \vc \; (\DC{Nil}\;\va) \; (\DC{Nil}\;\va) \; 
\cq\;\DC{Nil}\;\va \\
\RW{var}\;\AR{
\va\Hab\Type,\;\vk\Hab\Nat,\;\vc\Hab\TC{Num}\;\va,\\
\vx\Hab\va,\;\vxs\Hab\Vect\;\va\;\vk,\;
\vy\Hab\va,\;\vys\Hab\Vect\;\va\;\vk
\SC
\\
\FN{vAdd} \; \va \; (\suc\;\vk) \; \vc 
  \; ((\DC{::})\;\va\;\vk\;\vx\;\vxs) 
  \; ((\DC{::})\;\va\;\vk\;\vy\;\vys) 
   \\
   \hg\hg\cq\:((\DC{::})\;\va\;\vk\;((+)\;\vc\;\vx\;\vy)\; (\FN{vAdd}\;\va\;\vk\;\vc\;\vxs\;\vys))
}
}
}

\subsubsection{An Observation: Programming vs Theorem Proving}

\Idris{} programs may contain several high level constructs not present in \TT{}, such
as implicit arguments, type classes, \texttt{where} clauses, pattern matching \texttt{let}
and \texttt{case} constructs. We would like the high level language to be as expressive
as possible, while remaining possible to translate to \TT{}.

Before considering how to achieve this, we make an observation about the distinction between
programming and theorem proving with dependent types, and appropriate mechanisms for
constructing programs and theorems:

\begin{itemize}
\item \remph{Pattern matching} is a convenient abstraction for humans to write
programs, in that it allows a programmer to express exactly the computational
behaviour of a function.
\item \remph{Tactics}, such as those used in the Coq theorem
prover~\cite{Bertot2004}, are a convenient abstraction for building proofs and
programs by \remph{refinement}.
\end{itemize}

The idea behind the \Idris{} elaborator, therefore, is to use the high level
program to direct \demph{tactics} to build \TT{} programs by refinement.  The
elaborator is implemented as a Haskell monad capturing proof state, with a
collection of tactics for updating and refining the proof state.  The remainder
of this paper describes this elaborator and demonstrates how it is used to
implement the high level features of \Idris{}.





%\section{Proof State}

\subsection{Tactics}

\subsection{Unification}

\subsection{The Elaboration DSL}


\newcommand{\ttinterp}[1]{\mathcal{E}\interp{#1}}
\newcommand{\uninterp}[1]{\mathcal{T}\interp{#1}}

\section{Elaborating \Idris{}}

\label{sect:elaboration}

An \Idris{} program consists of a series of declarations --- data types,
functions, type classes and instances. In this section, we describe how these
high level declarations are translated into a \TT{} program consisting of
inductive families and pattern matching function definitions. We will need to
work at the \remph{declaration} level, and at the \remph{expression} level,
defining the following meta-operations.

\begin{itemize}
\item $\ttinterp{\cdot}$, which builds a \TT{} expression from an \Idris{} expression.
\item $\MO{Elab}$, which processes a top level \Idris{} declaration by generating
one or more \TT{} declarations.
\item $\MO{TTDecl}$, which adds a top level \TT{} declaration.
\end{itemize}


\subsection{The Development Calculus \TTdev}

\TT{} expressions are built by using high level \Idris{} expressions to
direct a tactic based theorem prover, which builds the \TT{} expressions
step by step, by refinement. In order to build expressions in this way,
the type theory needs to support
\remph{incomplete} terms, and a method for term construction. 
To achieve this, we extend \TT{} with \remph{holes},
calling the extended calculus \TTdev{}.
Holes stand for the parts of programs which have not yet been
instantiated; this largely follows the \Oleg{} development
calculus~\cite{McBride1999}.

The basic idea is to extend the syntax for binders with a \remph{hole}
binding and a \remph{guess} binding. 
These extensions are given in Figure \ref{ttdev}.
The \remph{guess} binding is
similar to a $\LET$ binding, but without any computational force,
i.e. there are no reduction rules for guess bindings. 
Using binders to represent holes is useful in a dependently typed setting since
one value may determine another. Attaching a guess to a binder ensures that
instantiating one such value also instantiates all of its dependencies. The
typing rules for binders ensure that no $?$ bindings leak into types.

\FFIG{
\AR{
\vb ::= \ldots 
 \:\mid\: \hole{\vx}{\vt} \:\:(\mbox{hole binding}) \:\:
 \:\mid\: \guess{\vx}{\vt}{\vt} \:\:(\mbox{guess})
\medskip\\
\Rule{
\Gamma;\hole{\vx}{\vS}\proves\ve\Hab\vT
}
{
\Gamma\proves\hole{\vx}{\vS}\SC\ve\Hab\vT
}
\hspace*{0.1cm}\vx\not\in\vT
\hspace*{0.1in}\mathsf{Hole}
\hg
\Rule{
\Gamma;\guess{\vx}{\vS}{\ve_1}\proves\ve_2\Hab\vT
}
{
\Gamma\proves\guess{\vx}{\vS}{\ve_1}\SC\ve_2\Hab\vT
}
\hspace*{0.1cm}\vx\not\in\vT
\hspace*{0.1in}\mathsf{Guess}
}
}
{\TTdev{} extensions}
{ttdev}


\subsection{Proof State}

\label{sect:prfstate}

A proof state is a tuple, $(\vC, \Delta, \ve, \vQ)$, containing:

\begin{itemize}
\item A global context, $\vC$, containing pattern matching definitions and their types
\item A local context, $\Delta$, containing pattern bindings
\item A proof term, $\ve$, in \TTdev{}
\item A hole queue, $\vQ$
%\item \remph{Deferred} definitions, $\vD$, for introducing global metavariables
\end{itemize}

The \remph{hole queue} is a priority queue of names of hole and guess binders 
$\langle\vx_1,\vx_2,\ldots,\vx_n\rangle$
in the proof term ---
ensuring that each bound name is unique. Holes refer to \remph{sub goals}
in the proof.
When this queue is empty, the proof term is complete.
Creating a \TT{} expression from an \Idris{} expression involves creating
a new proof state, with an empty proof term, and using the high level definition
to direct the building of a final proof state, with a complete proof term.

In the implementation, the proof state is captured in an elaboration monad,
\texttt{Elab}, which includes various operations for querying and updating
the proof state, manipulating terms, generating fresh names, etc. However, we will
describe \Idris{} elaboration in terms of meta-operations on the proof state,
in order to capture the essence of the elaboration process without being distracted
by implementation details. These meta-operations include: 

\begin{itemize}
\item \demph{Queries} which retrieve values from the proof state, without modifying
the state. For example, we can:
\begin{itemize}
\item Get the type of the current sub goal ($\MO{Type}$)
\item Retrieve the local context $\Gamma$ at the current sub goal ($\MO{Context}$)
\item Type check ($\MO{Check}$) or normalise ($\MO{Normalise}$) a term relative to $\Gamma$
\end{itemize}
\item \demph{Unification}, which unifies two terms (potentially solving sub goals) 
relative to $\Gamma$
\item \demph{Tactics} which update the proof term. Tactics operate on the sub term
at the binder specified by the head of the hole queue $\vQ$.
\item \demph{Focussing} on a specific sub goal, which brings a different sub goal to the
head of the hole queue.
%\item \demph{Deferring} a sub goal, which adds a new definition to the global context
%$\vC$ which solves the sub goal.
\end{itemize}

Elaboration of an \Idris{} expression involves creating a new proof state, running
a series of tactics to build a complete proof term, then retrieving and \remph{rechecking}
the final proof term, which must be a \TT{} program (i.e. does not contain any of the
\TTdev{} extensions). We call a sub-term which contains no hole or guess bindings 
\demph{pure}. Although a pure term does not contain hole or guess bindings, it may
nevertheless \remph{refer} to hole- or guess-bound variables.

The proof state is initialised with the $\MO{NewProof}$ operation. Given a global
context $\vC$, $\MO{NewProof}\:\vt$ sets up the proof state as:

\DM{
(\vC, \cdot, \hole{\vx}{\vt}\SC\vx, \langle\vx\rangle)
}

The local context is initially empty, and the initial hole queue is the $\vx$ standing for
the entire expression. The proof term is reset with the $\MO{NewTerm}$ operation.
In an existing proof state $(\vC, \Delta, \ve, \vQ)$,
$\MO{NewTerm}\:\vt$ discards the proof term and hole queue, and
updates the proof state to:

\DM{
(\vC, \Delta, \hole{\vx}{\vt}\SC\vx, \langle\vx\rangle)
}


This allows the elaborator to use pattern bindings from the left
hand side of a pattern matching definition in the term on the right hand side.

\subsection{System State}

\label{sect:sysstate}

The system state is a tuple, $(\vC,\vA,\vI)$, containing:

\begin{itemize}
\item A global context, $\vC$, containing pattern matching definitions and their types
\item Implicit arguments, $\vA$, recording which arguments are implicit for each global name
\item Type class instances, $\vI$, containing dictionaries for type classes
\end{itemize}

In the implementation, the system state is captured in a monad, \texttt{Idris}, 
which also wraps the current proof state. This monad 
includes additional information such as syntax overloadings,
command line options, and optimisations, which do not concern us here. Elaboration
of expressions requires access to the system state in particular in order to expand
implicit arguments and resolve type classes. 

For each global name, $\vA$ records whether its arguments are explicit, implicit,
or type class constraints.  For example, recall the declaration
of \texttt{vAdd}:

\begin{SaveVerbatim}{vAddImpT}

vAdd : Num a => Vect a n -> Vect a n -> Vect a n

\end{SaveVerbatim}
\useverb{vAddImpT} 

\noindent
Written in full, and giving each argument an explicit name, we get the
type declaration:

\begin{SaveVerbatim}{vAddImpT}

vAdd : (a : _) -> (n : _) -> (c : Num a) -> 
       (xs : Vect a n) -> (ys : Vect a n) -> Vect a n

\end{SaveVerbatim}
\useverb{vAddImpT} 

\noindent
For \tFN{vAdd}, the state records that \texttt{a} and \texttt{n} are implicit, 
\texttt{c} is a constraint, and \texttt{xs} and \texttt{ys} are explicit. When
the elaborator encounters an application of \tFN{vAdd}, it knows that unless these arguments
are given explicitly, the application must be expanded.

\newcommand{\Check}{\MO{Check}_\Gamma}
\newcommand{\Eval}{\MO{Normalise}_\Gamma}
\newcommand{\Unify}{\MO{Unify}_\Gamma}

\subsection{Tactics}

% Meta-operations Check, Normalise, Unify 
In order to build \TT{} expressions from \Idris{} programs, we define a collection
of meta-operations for querying and modifying the proof state. Meta-operations
may have side-effects including failure, or updating the proof state. We have the following
primitive meta-operations:

\begin{itemize}
\item $\MO{Focus}\:\vn$, which moves $\vn$ to the head of the hole queue.
\item $\MO{Unfocus}$, which moves the current hole to the back of the hole queue.
\item $\Check\:\ve$, which type checks an expression $\ve$ relative to a context
$\Gamma$, returning its type.
$\MO{Check}$ will fail
if the expression is not well-typed.
\item $\Eval\:\ve$, which evaluates a well-typed expression $\ve$ relative to a context 
$\Gamma$, returning its normal form.
\item 
$\Unify\:\ve_1\:\ve_2$, 
which unifies $\ve_1$ and $\ve_2$ by finding the values with which holes must be instantiated
for $\ve_1$ and $\ve_2$ to be convertible relative to $\Gamma$
(i.e. for $\Gamma\proves\ve_1\converts\ve_2$ to hold). $\MO{Unify}$ will fail
if it cannot find such values. If successful, $\MO{Unify}$ will update the proof state.
\end{itemize}

\demph{Tactics} are meta-operations which operate on the sub-term given
by the hole at the head of the hole queue in the proof state. They take the following form:

\DM{
\PA{\A\A}{
\MO{Tactic}_\Gamma & \:\vec{\VV{args}} & \:\vt & \MoRet{\vt'}
}
}

A tactic takes a sequence of zero or more arguments $\vec{\VV{args}}$ followed
by the sub-term $\vt$ on which it is operating. It runs relative to a context
$\Gamma$ which contains all the bindings and pattern bindings in scope at that
point in the term. The sub-term $\vt$ will either be a hole binding
$\hole{\vx}{\vT}\SC\ve$ or a guess binding $\guess{\vx}{\vT}{\vv}\SC\ve$. The
tactic returns a new term $\vt'$ which can take any form, provided it is
well-typed, with a type convertible to the type of $\vt$. 
Tactics may also have the side effect of updating the proof state,
therefore we will describe tactics in a pseudo-code with $\RW{do}$ notation.

We define a set of primitive tactics: $\MO{Claim}$, $\MO{Fill}$ and $\MO{Solve}$
which are used to create and destroy holes; and $\MO{Lambda}$, $\MO{Pi}$, $\MO{Let}$
and $\MO{Attack}$ which are used to create binders.

\subsubsection{Creating and destroying holes}

The $\MO{Claim}$ tactic, given a name and a type, adds a new hole binding in
the scope of the current goal $\vx$, adding the new binding to the hole queue, but
keeping $\vx$ at the head:

\DM{
\PA{\A\A}{
\MO{Claim}_\Gamma & (\vy \Hab\vS) & (\hole{\vx}{\vT}\SC\ve) & 
   \MoRet{\RW{return}\:\hole{\vx}{\vT}\SC\hole{\vy}{\vS}\SC\ve} \\
}
}

An obvious difficulty is in ensuring that names are unique throughout a proof term.
The implementation ensures that any hole created by the $\MO{Claim}$ tactic
has a unique name by checking against existing names in scope and modifying
the name if necessary. In this paper, we will assume that all created names are fresh.

The $\MO{Fill}$ tactic, given a value $\vv$, attempts to solve the current goal
with $\vv$, creating a guess binding in its place. $\MO{Fill}$ attempts to
solve other holes by unifying the expected type of $\vx$ with the type of $\vv$:

\DM{
\PA{\A\A}{
\MO{Fill}_\Gamma & \vv & (\hole{\vx}{\vT}\SC\ve) & 
   \MoRet{\RW{do}\:\AR{
   \vT' \leftarrow \Check\:\vv\\
   \Unify\:\vT\:\vT'\\
   \RW{return}\:\guess{\vx}{\vT}{\vv'}\SC\ve}
   } \\
}
}

\noindent
For example, consider the following proof term:

\DM{
\AR{
\hole{\vA}{\Set}\SC\hole{\vk}{\Nat}\SC
\hole{\vx}{\vA}\SC\hole{\vxs}{\Vect\:\vA\:\vk}\SC
\\
\hole{\vys}{\Vect\:\vA\:(\suc\:\vk)}\SC\vys
}
}

\noindent
If $\vx$ is in focus (i.e., at the head of the hole queue) and the elaborator
attempts to
$\MO{Fill}$ it with an $\TC{Int}$ value $42$, we have:

\begin{itemize}
\item $\Check\:42\:\mq\:\TC{Int}$
\item Unifying $\TC{Int}$ with $\vA$ (the type of $\vx$) is only possible if
$\vA\:=\:\TC{Int}$, so solve $\vA$.
\end{itemize}

\noindent
Therefore the resulting proof term is:

\DM{
\AR{
\hole{\vk}{\Nat}\SC
\guess{\vx}{\TC{Int}}{42}\SC\hole{\vxs}{\Vect\:\TC{Int}\:\vk}\SC
\\
\hole{\vys}{\Vect\:\TC{Int}\:(\suc\:\vk)}\SC\vys
}
}

The $\MO{Solve}$ tactic operates on a guess binding. If the guess is \remph{pure}, i.e., it
is a \TT{} term containing no hole or guess bindings, then the value attached to
the guess is substituted into its scope:

\DM{
\PA{\A}{
\MO{Solve}_\Gamma & (\guess{\vx}{\vT}{\vv}\SC\ve) &
   \MoRet{\RW{return}\:\ve[\vv/\vx]\hg\mbox{(if $\MO{Pure}\:\vv$)}}
}
}

The two step process, with $\MO{Fill}$ followed by $\MO{Solve}$, allows the elaborator
to work safely with incomplete terms, since an incomplete guess binding has no 
computational force. Once a term is complete in a guess binding, it may be substituted into the 
scope of the binding safely.
In each of these tactics, if any step fails, or the term in focus does not take
the correct form (e.g. is not a guess in the case of $\MO{Solve}$ or not a hole
in the case of $\MO{Claim}$ and $\MO{Fill}$, the entire tactic fails. We can
handle failure using the $\MO{Try}$ tactic combinator:

\DM{
\PA{\A\A\A}{
\MO{Try}_\Gamma & \VV{t1} & \VV{t2} & \vt &
   \MoRet{\AR{\VV{t1}_\Gamma\:\vt\hg\mbox{(if $\VV{t1}$ succeeds)} \\
              \VV{t2}_\Gamma\:\vt\hg\mbox{(otherwise)}}}
}
}

We have a primitive tactic $\MO{Fail}$, which may be invoked by
any tactic which encounters a failure condition (for example, an unsolvable unification
problem) and is handled by $\MO{Try}$.

\subsubsection{Creating binders}

We also define primitive tactics for constructing binders. Creating a $\lambda$
binding requires that the goal normalises to a function type:

\DM{
\PA{\A\A}{
\MO{Lambda}_\Gamma & \vn & (\hole{\vx}{\vT}\SC\vx) &
 \MoRet{\RW{do}\:\AR{
   \all{\vy}{\vS}\SC\vT'\:\leftarrow\:\Eval\:\vT \\
   \RW{return}\:\lam{\vn}{\vS}\SC\hole{\vx}{\vT'[\vn/\vy]}\SC\vx
   }
   }
}
}

\noindent
Creating a $\forall$ binding requires that the goal is a $\TC{Set}$:

\DM{
\PA{\A\A}{
\MO{Pi}_\Gamma & (\vn\Hab\vS) & (\hole{\vx}{\Set}\SC\vx) &
 \MoRet{\RW{do}\:\AR{
   \Set\:\leftarrow\:\Check\:\vS\\
   \RW{return}\:\all{\vn}{\vS}\SC\hole{\vx}{\Set}\SC\vx
   }
   }
}
}

\noindent
Creating a $\LET$ binding requires a type and a value.

\DM{
\PA{\A\A}{
\MO{Let}_\Gamma & (\vn\Hab\vS\defq\vv) & (\hole{\vx}{\vT}\SC\vx) &
 \MoRet{\RW{do}\:\AR{
   \Set\:\leftarrow\:\Check\:\vS\\
   \vS'\:\leftarrow\:\Check\:\vv\\
   \Unify\:\vS\:\vS'\\
   \RW{return}\:\LET\:\vn\Hab\vS\defq\vv\SC\hole{\vx}{\vT}\SC\vx
   }
   }
}
}

Each of these tactics require the term in focus to be of the form $\hole{\vx}{\vT}\SC\vx$.
This is important, because if the scope of the binding were an arbitrary expression $\ve$,
the binder would be scoped across this \emph{whole} expression rather than the subexpression
$\vx$ as intended.
The $\MO{attack}$ tactic ensures that a hole
is in the appropriate form, creating a new hole $\vh$ which is placed at the head
of the queue:

\DM{
\PA{\A}{
\MO{Attack}_\Gamma & (\hole{\vx}{\vT}\SC\ve) &
 \MoRet{\RW{return}\:\guess{\vx}{\vT}{(\hole{\vh}{\vT}\SC\vh)}\SC\ve}
}
}

Finally, we can convert a hole binding to a pattern binding by giving the 
pattern variable a name. This solves a hole
by adding the pattern binding to the proof state, and updating the proof term
with the pattern variable directly:

\DM{
\PA{\A\A}{
\MO{Pat}_\Gamma & \vn & (\hole{\vx}{\vT}\SC\ve) &
  \MoRet{\RW{do}\:\AR{
    \MO{PatBind}\:(\vx\Hab\vT)\\
    \RW{return}\:\ve[\vn/\vx]
  }}
}
}

The $\MO{PatBind}$ operation simply updates the proof state with the given pattern
binding. Once we have created bindings from the left hand side of a pattern
matching definition, for example, we can retain these bindings for use when
building the right hand side.

\subsubsection{Example}

Tactics are executed by a higher level meta-operation $\MO{RunTac}$, which
locates the appropriate sub-term, applies the tactic with the context
local to this sub-term, and
replaces the sub-term with the term returned by the
tactic. It then updates the hole queue in the proof state, and updates holes which have
been solved by unification. If the tactic creates new holes, these are automatically
added to the \remph{head} of the hole queue.
For example, consider the following simple
\TT{} definition for the identify function:

\DM{
\AR{
\FN{id}\Hab\all{\vA}{\Set}\SC\all{\va}{\vA}\SC\vA \\
\FN{id}\:=\:\lam{\vA}{\Set}\SC\lam{\va}{\vA}\SC\va  
}
}

We can build $\FN{id}$ either as a complete term, or by applying a sequence of tactics.
To achieve this, we create a proof state initialised with the type of $\FN{id}$ and
apply a series of $\MO{Lambda}$ and $\MO{Fill}$ operations using $\MO{RunTac}$.
Note that the types on each $\MO{Lambda}$ are solved by unification:

\DM{
\AR{
\MO{MkId}\:\mq\:\RW{do}\:
 \AR{
   \MO{NewProof}\:\all{\vA}{\Set}\SC\all{\va}{\vA}\SC\vA \\
   \MO{RunTac}\;\MO{Attack} \\
   \MO{RunTac}\;(\MO{Lambda}\;\vA) \\
   \MO{RunTac}\;\MO{Attack} \\
   \MO{RunTac}\;(\MO{Lambda}\;\va) \\
   \MO{RunTac}\;(\MO{Fill}\;\va)\\ 
   \MO{RunTac}\;\MO{Solve}\\
   \MO{RunTac}\;\MO{Solve}\\
   \MO{RunTac}\;\MO{Solve}\\
   \MO{Term}
 }
}
}

To aid readability, we will elide $\MO{RunTac}$, and use a semi-colon to indicate
sequencing --- in the implementation we use wrapper functions for each tactic to
apply $\MO{RunTac}$.
Using this convention, we can build $\FN{id}$'s type and definition as shown
in Figure \ref{idelab}. Note that $\MO{Term}$ retrieves the proof term from the current proof
state. Both $\MO{MkIdType}$ and $\MO{MkId}$ finish by returning a completed \TT{} term.
Note in particular that each $\MO{Attack}$ and each $\MO{Fill}$, which create new guesses,
are closed with a $\MO{Solve}$.

Setting up elaboration in this way, with a proof state captured in a monad,
and a primitive collection of tactics,
makes it easy to derive more complex tactics for elaborating higher level language constructs,
in much the same way as the \texttt{Ltac} language in Coq. As a result, the
description of elaboration of
a language construct (or a program such as $\FN{id}$) bears a strong resemblance to
a Coq proof script.

\FFIG{
\AR{
\MO{MkIdType}\:\mq\:\RW{do}\;
 \AR{
   \MO{NewProof}\:\Set\\
   \MO{Attack} ; \MO{Pi}\:(\vA\Hab\Set) ;
   \MO{Attack} ; \MO{Pi}\:(\va\Hab\vA) \\
   \MO{Fill}\;\vA \\
   \MO{Solve} ; \MO{Solve} ; \MO{Solve} \\
   \MO{Term}
 }
 \medskip\\
\MO{MkId}\:\mq\:\RW{do}\;
 \AR{
   \vt\:\leftarrow\:\MO{MkIdType}; \MO{NewProof}\:\vt\\
   \MO{Attack} ; \MO{Lambda}\;\vA ; 
   \MO{Attack} ; \MO{Lambda}\;\va \\
   \MO{Fill}\;\va \\
   \MO{Solve} ; \MO{Solve} ; \MO{Solve}\\
   \MO{Term}
 }
}
}
{Building $\FN{id}$ with tactics}
{idelab}

%--- give unify in full, esp. as it solves sub goals? Maybe...

% Unify' G x t             = Success (x, t) if ?x : t in G
% Unify' G t x             = Success (x, t) if ?x : t in G
% Unify' G (b x. e) (b' x'. e')   = Unify' G b b'; Unify' G;b e e'[x/x']
% Unify' G ((\x.e) x) e'   = Unify' G e e' 
% Unify' G e ((\x.e') x)   = Unify' G e e' 
% Unify' G (f es) (f' es') = vs <- Unify' G f f'; Injective f
                             
% Unify' G x y             = Success () if G |- x == y
% Unify' G . .             = Failure

% Unify' G (\x : t . e) (\x : t' . e') = Unify' G t t'; Unify' G e e'
% ...


%\DM{
%}

\subsection{Elaborating Expressions}

\newcommand{\piimp}[2]{\mbox{\texttt{\{ $#1$ : $#2$ \} -> }}}
%\mathtt{\{}#1\mathtt{:}#2\mbox{\texttt{\} -> }}}
\newcommand{\piexp}[2]{\mbox{\texttt{( $#1$ : $#2$ ) -> }}}
%\newcommand{\piexp}[2]{\mathtt{(}#1\mathtt{:}#2\mbox{\texttt{) -> }}}
\newcommand{\piconst}[1]{\mbox{\texttt{$#1$ => }}}
\newcommand{\icase}{\mathtt{case}}
\newcommand{\iwith}{\mathtt{with}}
\newcommand{\idata}{\mathtt{data}}
\newcommand{\iclass}{\mathtt{class}}
\newcommand{\iinstance}{\mathtt{instance}}
\newcommand{\iwhere}{\mathtt{where}}
\newcommand{\iof}{\mathtt{of}}
\newcommand{\ilet}[2]{\mathtt{let}\;#1\;\mathtt{=}\;#2\;\mathtt{in}}
\newcommand{\ilam}[1]{\mathtt{\backslash}\;#1\;\mathtt{=>}}
\newcommand{\iarg}[2]{\mbox{\texttt{\{$#1$ = $#2$\}}}}
\newcommand{\ihab}[2]{\mbox{\texttt{$#1$ : $#2$}}}
\newcommand{\carg}[1]{\mbox{\texttt{\{\{$#1$\}\}}}}
\newcommand{\fatarrow}{\mbox{\texttt{=>}}}
\newcommand{\ibar}{\mbox{\texttt{|}}}
\newcommand{\mvar}[1]{\mbox{\texttt{?}}#1}

\FFIG{
\AR{
\begin{array}{rll@{\hg}rll}
\ve ::= & \vc & (\mbox{constant}) &
\mid & \vx & (\mbox{variable}) \\
\mid & \vt & (\mbox{type}) &
\mid & \ve\:\ta & (\mbox{function application}) \\
\mid & \ilam{\vx}\:\ve & (\mbox{lambda abstraction}) &
\mid & \ilet{\vx}{\ve}\:\ve & (\mbox{let binding}) \\
%\mid & \icase\:\ve\:\iof\:\vec{\VV{alt}} & (\mbox{case expression}) &
%\mid & \mvar{\vx} & (\mbox{metavariable}) \\
\mid & \_ & (\mbox{placeholder})
\end{array}
\medskip\\
\begin{array}{rll}
\va :: = & \ve & (\mbox{normal argument}) \\
\mid & \iarg{\vx}{\ve} & (\mbox{implicit argument with value}) \\
\mid & \carg{\ve} & (\mbox{explicit class instance}) 
\end{array}
\medskip\\
\begin{array}{rll}
\vt ::= & \ve & (\mbox{expression}) \\
\mid & \piexp{\vx}{\vt}\vt & (\mbox{explicit function space}) \\
\end{array}
\medskip\\
\begin{array}{rll}
\VV{declty} ::= & \ve & (\mbox{expression}) \\
\mid & \piexp{\vx}{\vt}\VV{declty} & (\mbox{explicit function space}) \\
\mid & \piimp{\vx}{\vt}\VV{declty} & (\mbox{implicit function space}) \\
\mid & \VV{constr}\;\VV{declty} & (\mbox{constrained type}) \\
\end{array}
\medskip\\
\begin{array}{rll}
\VV{constr} ::= & \piconst{\ttt} & (\mbox{type class constraint}) \\
\end{array}
}
}
{\IdrisM{} expressions}
{idrism}

\FFIG{
\AR{
\begin{array}{rll}
\vd ::= & \ihab{\vx}{\VV{declty}} & (\mbox{type declaration}) \\
\mid & \VV{pclause} & (\mbox{pattern clause}) \\
\mid & \VV{ddecl} & (\mbox{data type declaration}) \\
\mid & \VV{cdecl} & (\mbox{class declaration}) \\
\mid & \VV{idecl} & (\mbox{instance declaration}) \\
\end{array}
\medskip\\
\begin{array}{ll}
\begin{array}{rll}
\VV{pclause} ::= & 
\vx\:\ttt\:[\ibar\;\te]\mbox{\texttt{ = }}\ve \hg[\iwhere\;\td]\\ 
\mid & \AR{
\vx\:\ttt\:[\ibar\;\te]\;\iwith\;\ve\\
\hg\vec{\VV{pclause}}
}
\end{array}
&
\begin{array}{rll}
\VV{ddecl} ::= & \idata\;\ihab{\vx}{\VV{declty}}\;\iwhere\;\vec{\VV{con}}\\
\VV{con} ::= & \ihab{\vx}{\VV{declty}}\\
\medskip\\
\VV{cdecl} ::= & \iclass\;[\VV{constr}]\;\vx\;(\ihab{\tx}{\ttt})\;\iwhere\;\td
\\
\VV{idecl} ::= & \iinstance\:[\VV{constr}]\;\vx\;\ttt\;\iwhere\;\td
\end{array}
\end{array}
}
}
{\IdrisM{} declarations}
{idrismd}

We define a language \IdrisM{}, with expression syntax given in Figure \ref{idrism}
and declaration syntax given in Figure \ref{idrismd}.
\IdrisM{} is a subset of \Idris{} without syntactic sugar --- that is, without
tuple syntax, \texttt{do}-notation or infix operators --- and with implicit
arguments in types bound explicitly 
(e.g. $\piimp{\va}{\_}\piimp{\vn}{\_}\TC{Vect}\:\va\:\vn$
instead of simply $\Vect\:\va\:\vn$).
Note that we separate implicit and type constraint bindings, to ensure that they
only appear in top level declarations.
It is in general straightforward to
convert full \Idris{} to \IdrisM{} --- syntactic sugar is implemented by a
direct source transformation, and implicit arguments can be identified as the names
which are free in a type in a non-function position.
%
\IdrisM{} differs from \TT{} in several important respects. It has implicit
syntax and type classes, and functions are applied to multiple arguments rather
than one at a time. 

%Extensions over \TT{}: implicit syntax, case expressions. Functions are applied to
%multiple arguments, rather than one at a time (this helps with implicit syntax).
%\IdrisM{} also supports explicit \demph{metavariables}. A metavariable \texttt{?mvar}
%creates a new global function \tFN{mvar}, with its type such that it would
%be type correct to apply \tFN{mvar} to all of the variables in scope.

%Implicit and type class arguments? Expanded at the application site (we need to know
%it's the global name after all and we do that by type).

To elaborate expressions, we define a meta-operation $\ttinterp{\cdot}$ which
runs relative to a proof state (see Section \ref{sect:prfstate}).  Its effect is
to update the proof state so that the hole in focus contains a representation
of the given expression, by applying tactics. We assume that the proof state
has already been set up, which means that elaboration can always be
\remph{type-directed} since the proof state contains the type of the expression we
are building.

\subsubsection{Elaborating variables and constants}

In the simplest cases, there is a direct translation from an \IdrisM{} expression to
a \TT{} expression --- we build \TT{} representations of variables and constants using
the \MO{Fill} tactic:

\DM{
\AR{
\ttinterp{\vx}\:\mq\:\RW{do}\:\MO{Fill}\:\vx;\:\MO{Solve}\\
\ttinterp{\vc}\:\mq\:\RW{do}\:\MO{Fill}\:\vc;\:\MO{Solve}\\
}
}

We need not concern ourselves with type checking variables or constants here --- $\MO{Fill}$
will handle this, type checking $\vx$ or $\vc$ and unifying the result with the
hole type. If there are any errors, elaboration will fail.

If we are building the left hand side of a pattern clause, however, there is a problem,
as it is the left hand side which \remph{defines} variables. In this context, we assume
that attempting to elaborate a variable which does not type check means that the variable
is a pattern variable:

\DM{
\ttinterp{\vx}\:\mq\:
\MO{Try}\:\AR{(\RW{do}\:\MO{Fill}\:\vx;\:\MO{Solve})\\
(\MO{Pat}\:\vx)\hg\hg\mbox{(If elaborating a left hand side)}}
}

We also need to elaborate \remph{placeholders}, which are subexpressions we expect to
solve by unification. In this case, we simply move on to the next hole in the queue, moving
the current hole to the end of the queue with $\MO{Unfocus}$:

\DM{
\ttinterp{\_}\:\mq\:\MO{Unfocus}
}

On encountering a placeholder, our assumption is that unification will eventually solve
the hole. At the end of elaboration, any holes remaining unsolved on the left hand side
become pattern variables. If there are any unsolved on the right hand side, elaboration
fails.

\subsubsection{Elaborating bindings}

To elaborate a $\lambda$-binding, we $\MO{Attack}$ the hole, which must be a function type,
apply the $\MO{Lambda}$ tactic, elaborate the scope, then $\MO{Solve}$, which discharges
the $\MO{Attack}$:

\DM{
\ttinterp{\ilam{\vx}\ve}\:\mq\:
\RW{do}\:
\AR{
\MO{Attack};\;\MO{Lambda}\:\vx\\
\ttinterp{\ve}\\
\MO{Solve}
}
}

Note that there is no type on the $\lambda$-binding in \IdrisM{}. There is no need --- since
elaboration is type directed, the $\MO{Lambda}$ tactic finds the type of the binding by
looking at the type of the hole. In full \Idris{}, types are allowed on bindings, and the
elaborator merely checks that the given type is equivalent to the inferred type.

Elaborating a function type is more tricky, since we have to elaborate the argument
type (itself an \IdrisM{} expression) then elaborate the scope. To achieve this, we
create a new goal $\vX$ for the argument type $\vt$, where $\vX$ is a fresh name,
and introduce a function binding with argument type $\vX$. We can then focus on
$\vX$, and elaborate it with the expression $\vt$. Finally, we elaborate the
scope of the binding.

\DM{
\AR{
\ttinterp{\piexp{\vx}{\vt}\ve}\:\mq\:
\RW{do}\:
\AR{
\MO{Attack};\;\MO{Claim}\:(\vX\Hab\Set)\\
\MO{Pi}\:(\vx\Hab\vX)\\
\MO{Focus}\:\vX\\
\ttinterp{\vt};\;
\ttinterp{\ve}\\
\MO{Solve}
}
}
}

Elaborating $\vt$ will involve solving unification problems, which will, if the
program is type correct, solve $\vX$. After focussing on $\vX$ and elaborating
$\vt$, there is no need to refocus on the hole representing the scope, as it was
previously at the head of the hole queue before focussing on $\vX$.
Elaboration of implicit and constraint
argument types is exactly the same --- \TT{} makes no distinction between then.

To elaborate a \texttt{let} binding, we take a similar approach, creating new subgoals
for the \texttt{let}-bound value and its type, then elaborating the scope. Again,
if elaborating the scope is successful, unification will solve the claimed variables
$\vV$ and $\vX$.

\DM{
\ttinterp{\ilet{\vx}{\vv}\:{\ve}}\:\mq\:\RW{do}\:
\AR{
\MO{Attack};\;
\MO{Claim}\:(\vX\Hab\Set);\; 
\MO{Claim}\:(\vV\Hab\vX)\\
\MO{Let}\:(\vx\Hab\vX\defq\vV)\\
\MO{Focus}\:\vV\\
\ttinterp{\vv};\;
\ttinterp{\ve}
}
}
\subsubsection{Elaborating applications}

There are two cases to consider when elaborating applications:

\begin{itemize}
\item Applying a global function name to arguments, some of which may be implicit.
\item Applying an expression, which does not have implicit arguments, to an argument.
\end{itemize}

In the first case, the elaborator must expand the application to include implicit arguments.
For example \texttt{vAdd xs ys} is expanded to
\texttt{vAdd \{a=\_\} \{n=\_\} \{\{\_\}\} xs ys}, adding the implicit arguments
\texttt{a} and \texttt{n} and a type class instance argument. The meta-operation
$\MO{Expand}\:\vx\:\ta$, given a global function name $\vx$ and the
arguments supplied by the programmer $\ta$, returns a new argument list
$\ta'$ with implicit and type class arguments added. Each value in
$\ta'$ is paired with an explicit name for the argument.

Implicit arguments are solved by unification
--- the type or value of another argument determines the value of an implicit argument,
with the appeal to $\MO{Unify}$ in the $\MO{Fill}$ tactic solving as many extra holes
as it can. However, unification problems can take several forms. For example, assuming
$\vf$, $\vg$, $\vx$ are holes, we might have unification problems of the following
forms:

\DM{
\AR{
\MO{Unify}_\Gamma\;\vf\:\TC{Int}\\
\MO{Unify}_\Gamma\;(\vf\:\vx)\:\TC{Int}\\
\MO{Unify}_\Gamma\;(\vf\:\vx)\:(\vg\:\vx)\\
}
}

The first problem has a solution: $\vf=\TC{Int}$. The second and third problems have
no solution without further information. In the second case, we cannot conclude 
anything about $\vf$ or $\vx$ just from knowing that $\vf\:\vx\:=\:\TC{Int}$.
In the third case, although we have $\vx\:=\:\vx$ in argument position, we cannot
conclude that $\vf\:=\:\vg$ unless we know that the function which solves $\vf$ or
$\vg$ is injective.

Once $\vf$ or $\vg$ is solved in some other way, perhaps by being given explicitly
in another argument, these unification problems can make progress. Otherwise, unification
fails. As a result, the order in which functions and arguments are
elaborated matters. Sometimes, we may learn more by elaborating a function first
and the arguments later, sometimes it may be the other way around. 

One option when unification fails is to store the problem (i.e. the expressions to be
unified with their local context) and refine it with further information when it becomes 
available. In practice, we have found a simpler approach to be effective: try
elaborating the function first, then arguments. If that fails, try elaborating the
arguments first.

Elaborating a global function application makes a $\MO{Claim}$ for each argument
in turn. Then the function is elaborated, applied to each of the claimed arguments.
Finally, each argument which has been given explicitly is elaborated. If this fails,
elaboration tries again in the opposite order. Finally, any type
class instance arguments are resolved with the built-in $\MO{Instance}$ tactic.

\DM{
\ttinterp{\vx\:\ta}\:\mq\:
\AR{
\RW{do}\:
\AR{
(\tn,\tv)\:\gets\:\MO{Expand}\:\vx\:\ta\\
\vec{\MO{Claim}}\:(\tn\Hab\tT)\\
\MO{Try}\:\AR{
(\RW{do}\:\AR{
\MO{Fill}\:\vx\:\tn\\
\vec{\MO{Focus}}\:\tn;\:\ttinterp{\tv}\hg\mbox{(for non-placeholder $\vv$)}\\
\MO{Solve})
}\\
(\RW{do}\:\AR{
\vec{\MO{Focus}}\:\tn;\:\ttinterp{\tv}\hg\mbox{(for non-placeholder $\vv$)}\\
\MO{Fill}\:\vx\:\tn\\
\MO{Solve})
}}
\\
\vec{\MO{Instance}}\:\tn\hg\mbox{(for type class constraint argument $\vn$)}
}
}
}

The $\MO{Instance}$ tactic focuses on the given hole and searches the context for
a type class instance which would solve the goal directly. First, it examines the local
context, then recursively searches the global context. $\MO{Instance}$ is covered
in detail in Section \ref{sect:instance}.

To elaborate a simple function application, of an arbitrary expression to an arbitrary
argument, we need not worry about implicit arguments or class constraints. Instead,
the function and argument are elaborated, and the results applied. Since elaboration
is type directed, however, there must be an appropriate type for the function.
As before, elaboration tries two orders, as it may learn something from elaborating
the function or argument which helps with unification:

\DM{
\ttinterp{\ve\:\va}\:\mq\:\RW{do}\:
\AR{
\MO{Claim}\:(\vA\Hab\Set);\;\MO{Claim}\:(\vB\Hab\Set)\\
\MO{Claim}\:(\vf\Hab\vA\to\vB);\;\MO{Claim}\:(\vs\Hab\vA)\\
\MO{Try}\:
\AR{
(\RW{do}\:\AR{
\MO{Focus}\:\vf;\;\ttinterp{\ve}\\
\MO{Focus}\:\vs;\;\ttinterp{\va})}\\
(\RW{do}\:\AR{
\MO{Focus}\:\vs;\;\ttinterp{\va}\\
\MO{Focus}\:\vf;\;\ttinterp{\ve})}\\
}
}
}

%\subsubsection{Elaborating metavariables}

%\subsubsection{Elaborating \texttt{case} expressions}

\subsection{Elaborating Declarations}

To elaborate declarations, we define a meta-operation $\MO{Elab}$, which runs
relative to the system state (see Section \ref{sect:sysstate}) and has access to the
current proof state. $\MO{Elab}$ uses $\ttinterp{\cdot}$ to help translate
\IdrisM{} type, function and class declarations into \TT{} declarations.

\newcommand{\edo}[1]{\RW{do}\:\AR{#1}}

\subsubsection{Elaborating Type Declarations}

Elaborating a type declaration involves creating a new proof state,
translating the \IdrisM{} type to a \TT{} type, then adding the resulting type
as a \TT{} declaration:

\DM{
\MO{Elab}\:(\vx\Hab\vt)\:\mq
\:\edo{\MO{NewProof}\:\Set\\
       \ttinterp{\vt}\\
       \vt'\gets\MO{Term}\\
       \MO{TTDecl}\:(\vx\Hab\vt')
}}

The final $\MO{TTDecl}$ takes the result of elaboration, type checks it,
and adds it to the global context if type checking succeeds. This final type check
ensures that the elaboration process does not allow any ill-typed terms to creep
into the context.

\subsubsection{Elaborating Data Types}

Elaborating a data declaration involves elaborating the type declaration itself,
as a normal type declaration. This ensures that the type is in scope when
elaborating the constructor types. Then using the results, the elaborator
adds a \TT{} data declaration.

\DM{
\AR{
\MO{Elab}\:(\idata\;\ihab{\vx}{\vt}\;\iwhere\;\tc)\\
\hg\hg\mq\:\edo{\MO{NewProof}\:\Set\\
                \ttinterp{\vt}\\
                \vt'\gets\MO{Term}\\
                \MO{TTDecl}\:(\vx\Hab\vt')\\
                \tc'\gets\vec{\MO{ElabCon}}\:\tc\\
                \MO{TTDecl}\:(\Data\:\vx\Hab\vt'\:\Where\:\tc')
}
\AR{
\MO{ElabCon}\:(\vx\Hab\vt)\\
\hg\hg\mq\: \edo{\MO{NewProof}\:\Set\\
                 \ttinterp{\vt}\\
                 \vt'\gets\MO{Term}\\
                 \RW{return}\:(\vx\Hab\vt')
}
}}}

\subsubsection{Elaborating Pattern Matching}

Elaborating a pattern matching definition works clause by clause, elaborating
the left and right hand sides in turn. $\MO{ElabClause}$ returns the elaborated
left and right hand sides, and may have the side effect of adding entries
to the global context, such as definitions in $\iwhere$ clauses.
First, let us consider the simplest case, with no $\iwhere$ clause:

\DM{
\MO{ElabClause}\:(\vx\:\ttt\:=\:\ve)\:\mq\:?
}

How does the left hand side get elaborated, given that elaboration is type directed, and
its type is not known until after elaboration? 
This can be achieved without any change to the elaborator by defining 
a type $\TC{Infer}$:

\DM{
\Data\hg\TC{Infer}\Hab\Set\hg\Where\hg\DC{MkInfer}\Hab\all{\va}{\Set}\SC\va\to\TC{Infer}
}

Now elaboration of the left hand side proceeds by elaborating $\DC{MkInfer}\:\_\:(\vx\:\ttt)$ and
extracting the value and unified type when complete.

\DM{
\AR{
\MO{ElabClause}\:(\vx\:\ttt\:=\:\ve)\:\mq\:\\
\hg\hg\edo{
\MO{NewProof}\:\TC{Infer};\;
\ttinterp{\DC{MkInfer}\:\_\:(\vx\:\ttt)}\\
\DC{MkInfer}\:\vT\:\VV{lhs}\gets\MO{Term}\\
\tp\gets\MO{Patterns}\\
\MO{NewTerm}\:\vT;\;
\ttinterp{\ve}\\
\VV{rhs}\gets\MO{Term}\\
\RW{return}\:(\RW{var}\:\tp\SC\VV{lhs}\:=\:\VV{rhs})
}
}
}

This infers a type for the left hand side, creating pattern variable bindings. Then, 
it elaborates the right hand side using the inferred type of the left hand side
and the inferred pattern bindings, which are retrieved from the state with the
$\MO{Patterns}$ operation. Finally, it returns a pattern clause in \TT{} form.
Elaborating a collection of pattern clauses then proceeds by mapping $\MO{ElabClause}$ over
the clauses and adding the resulting collection to \TT{}.

\DM{
\AR{
\MO{Elab}\:\vec{\VV{pclause}}\:\mq\:
\edo{
\tc\gets\vec{\MO{ElabClause}}\:\vec{\VV{pclause}}\\
\vec{\MO{TTDecl}}\:\tc
}
}
}

Elaborating a clause is made only slightly more complex by the presence of
$\iwhere$ clauses. In this case, after elaborating the left hand side, the pattern
bound variables are added as extra arguments to the declarations in the $\iwhere$ block,
which are then recursively elaborated before the right hand side.

\DM{
\AR{
\MO{ElabClause}\:(\vx\:\ttt\:=\:\ve\:\iwhere\:\td)\:\mq\:\\
\hg\hg\edo{
\MO{NewProof}\:\TC{Infer};\;
\ttinterp{\DC{MkInfer}\:\_\:(\vx\:\ttt)}\\
\DC{MkInfer}\:\vT\:\VV{lhs}\gets\MO{Term}\\
\tp\gets\MO{Patterns}\\
(\td', \ve') \gets\MO{Lift}\:\tp\:\td\:\ve\\
\vec{\MO{Elab}}\:\td'\\
\MO{NewTerm}\:\vT;\;
\ttinterp{\ve'}\\
\VV{rhs}\gets\MO{Term}\\
\RW{return}\:(\RW{var}\:\tp\SC\VV{lhs}\:=\:\VV{rhs})
}
}
}

In \TT{}, all definitions must be at the top level. Therefore, the declarations 
in the $\iwhere$ block are lifted out, adding the pattern bound names as additional arguments
to ensure they are in scope, using the $\MO{Lift}$ operation. This also modifies
the right hand side $\ve$ to use the lifted definitions, rather than the original.

\subsubsection{Elaborating the \texttt{with} rule}

The \texttt{with} rule allows \remph{dependent} pattern matching on intermediate values.
Translating this into \TT{} involves constructing an auxiliary top level
definition for 
the intermediate pattern match. For example, recall \tFN{natToBin}:

\useverb{natToBin}

\noindent
This is equivalent to the following, using top level definitions only:

\begin{SaveVerbatim}{natToBinw}

natToBin : Nat -> List Bool
natToBin O = Nil
natToBin k = natToBin' k (parity k)

\end{SaveVerbatim}
\begin{SaveVerbatim}{natToBinwp}

natToBin' : (n : Nat) -> Parity n -> List Bool
natToBin' (j + j)     even = False :: natToBin j
natToBin' (S (j + j)) odd  = True  :: natToBin j

\end{SaveVerbatim}
\useverb{natToBinw}

\useverb{natToBinwp}

\noindent
Elaboration of \texttt{with} builds the type declaration and auxiliary function
automatically:


\DM{
\AR{
\MO{ElabClause}\:(\vx\:\ttt\:\iwith\:\ve\:\vec{\VV{pclause}})\:\mq\\
\hg\hg\edo{
\MO{NewProof}\:\TC{Infer};\;
\ttinterp{\DC{MkInfer}\:\_\:(\vx\:\ttt)}\\
\DC{MkInfer}\:\vT\:\VV{lhs}\gets\MO{Term}\\
\tp\gets\MO{Patterns}\\
\MO{NewTerm}\:\TC{Infer};\;
\ttinterp{\DC{MkInfer}\:\_\:\ve}\\
\DC{MkInfer}\:\vW\:\vw'\gets\MO{Term}\\
\MO{TTDecl}\:(\vx'\Hab\tp\to\vW\to\vT)\\
\tc'\gets\vec{\MO{MatchWith}}\:\tp\:(\vx\:\ttt)\:\vec{\VV{pclause}}\\
\vec{\MO{Elab}}\:\tc'\\
\RW{return}\:(\RW{var}\:\tp\SC\VV{lhs}\:=\:\vx'\:\tp\:\vw')
}
\medskip\\
\MO{MatchWith}\:\tp\:(\vx\:\ttt)\:(\vx\:\tw\:\mid\:\ve\:\VV{rhs})\:\mq\\
\hg\hg\edo{
\tp'\gets\vec{\MO{Match}}\:\tw\:\ttt\\
\MO{CheckComplete}\:\tp\:\tp'\\
\RW{return}\:(\vx'\:\tp'\:\ve\:\VV{rhs})
}
}
}

A $\iwith$ block in a clause for $\vx$
results in an auxiliary definition $\vx'$, where $\vx'$ is a fresh name,
which takes an extra argument corresponding to the intermediate result which
is to be matched.
$\MO{MatchWith}$ matches the left hand side of the top level definition against
each $\VV{pclause}$, using the result of the match to build each pattern clause
for $\vx'$. 
The clauses in the with block must be of a specific form: they must have an initial
set of arguments $\tw$ which matches the outer clause arguments
$\ttt$, and an extra argument
$\ve$ which is the same type as the scrutinee of the \texttt{with}, $\vW$.
The right hand side, $\VV{rhs}$, may either return a value directly, containing 
\texttt{where} clauses, or be a nested \texttt{with} block. In any case, it
remains unchanged.
$\MO{Match}\:\vw\:\vt$ returns a list of pattern bindings containing
the variables in $\vt$ and their matches in $\vw$.
If the \texttt{with} block is well-formed, then
the resulting set of patterns $\tp'$ contains exactly the same pattern variables
as $\tp$, which is verified by $\MO{CheckComplete}$.


\subsubsection{Elaborating Class and Instance Declarations}

Type classes are implemented as dictionaries of functions for each instance. Therefore,
a type class declaration elaborates to a record type containing all of the functions
in the class, and an instance is simply an instance of the record. The methods
are translated to top level functions which extract the relevant function from the 
dictionary.  
For example, for the \texttt{Show} type class, the \texttt{class}
declaration is translated to:

\begin{SaveVerbatim}{showdata}

data Show : Set -> Set where
    instanceShow : (show : a -> String) -> Show a

show : Show a -> a -> String
show (instanceShow show') = show'

\end{SaveVerbatim}
\useverb{showdata} 

\noindent
An instance for a type \texttt{a} is then a value of type \texttt{Show a}; for example for Nat:

\begin{SaveVerbatim}{shownatf}

showNat : Show Nat;
showNat = instanceShow show' where
    show' : Nat -> String
    show' O = "O"
    show' (S k) = "s" ++ show k

\end{SaveVerbatim}
\useverb{shownatf} 

The call to \tFN{show} in the body of the instance uses the top level \tFN{show}
function, with the type class instance resolved by the elaborator.
Where instances have constraints, these constraints are passed on to
the function declaration for the instance. For example, for \texttt{Show (List a)}:

\begin{SaveVerbatim}{showlista}

showList : Show a => Show (List a)
showList = instanceShow show' where
    show' : List a -> String
    show' []        = "[]"
    show' (x :: xs) = show x ++ " :: " ++ show xs

\end{SaveVerbatim}
\useverb{showlista} 

\noindent
Where classes have constraints, the
constraints are again passed on to the data type declaration, and resolved by the
elaborator. For example, for \texttt{Ord}:

\begin{SaveVerbatim}{eqorda}

class Eq a => Ord a where
    compare : a -> a -> Ordering
    max : a -> a -> a
    -- remaining methods elided

\end{SaveVerbatim}
\useverb{eqorda}

\noindent
This translates to:

\begin{SaveVerbatim}{orddata}

data Ord : Set -> Set where
    instanceOrd : Eq a => (compare : a -> a -> Ordering) -> 
                          (max : a -> a -> a) ->  Ord a 

\end{SaveVerbatim}

\begin{SaveVerbatim}{orddatab}
compare : Ord a -> a -> a -> Ordering
compare (instanceOrd compare' max') = compare'

max : Ord a -> a -> a -> Ordering
max (instanceOrd compare' max') = max'

\end{SaveVerbatim}
\useverb{orddata}

\useverb{orddatab}

\noindent
The elaborator also adds functions to retrieve parent classes from a dictionary, to assist
with type class resolution. In this case:

\begin{SaveVerbatim}{ordparent}

OrdEq : Ord a -> Eq a
OrdEq (instanceOrd {{eq}} compare' max') = eq

\end{SaveVerbatim}
\useverb{ordparent} 

\noindent
In general, \texttt{class} declarations are elaborated as follows:

\DM{
\AR{
\MO{Elab}\:(\iclass\;\piconst{\tc}\TC{C}\:(\ta\Hab\ttt)\:\iwhere\:\td)
\\
\hg\hg
\mq\:\RW{do}\:\AR{
(\vec{\VV{meth}}\Hab\ttt)\gets\td\\
\MO{Elab}\:
\AR{(\RW{data}\:\TC{C}\Hab(\ta\Hab\ttt)\to\Set\:\iwhere\\
\hg\DC{InstanceC}\Hab\piconst{\tc}\td\to\TC{C}\:\ta)}
}
}
}

\DM{
\AR{
\AR{
\vec{\MO{ElabMeth}}\:\vec{\VV{meth}}\:\td\\
\vec{\MO{ElabParent}}\:\tc
}
\medskip\\
\MO{ElabMeth}\:(\vf\Hab\vt)\\
\hg\hg\mq\:\RW{do}\:\AR{
\MO{Elab}\:(\vf\Hab\TC{C}\:\ta\to\vt)\\
\MO{Elab}\:(\vf\:(\DC{InstanceC}\:\vec{\VV{meth}})\:=\:\vec{\VV{meth}[\vf]})
}
\medskip\\
\MO{ElabParent}\:\vc\\
\hg\hg\mq\:\RW{do}\:\AR{
\MO{Elab}\:(\VV{cC}\Hab\TC{C}\:\ta\to\vc)\\
\MO{Elab}\:(\VV{cC}\:(\DC{InstanceC}\:\carg{\vc'}\:\vec{\VV{meth}})\:=\:\vc')
}
}
}

\noindent
Then \texttt{instance} declarations are elaborated as follows:

\DM{
\AR{
\MO{Elab}\:(\iinstance\;\piconst{\tc}\TC{C}\:\ttt\:\iwhere\:\td)\\
\hg\hg\mq\:\RW{do}\:\AR{
(\vec{\VV{meth}}, \vec{\VV{ty}})\gets\td\\
\vec{\VV{ty'}}\:=\:\vec{\VV{ty}}[\ttt/\ta]\\
\MO{Elab}\:(\FN{instanceC}\Hab\piconst{\tc}\TC{C}\:\ttt)\\
\MO{Elab}\:\AR{
(\FN{instanceC}\:=\:\DC{InstanceC}\;\vec{\VV{meth'}}\;\iwhere
\\
\hg\vec{\VV{meth'}}\Hab\vec{\VV{ty'}}\\
\hg\td
)}
}
}
}

Adding default definitions is straightforward: simply insert the default definition where
there is a method missing in an \texttt{instance} declaration.

\textbf{Remark:} Here, we have added a higher level language construct (type
classes) simply by elaborating in terms of lower level language constructs
(data types and functions).  We achieve type class resolution by implementing a
tactic, $\MO{Instance}$.  We can build several extensions this way because we
have effectively built, bottom up, an Embedded Domain Specific Language for
constructing programs in \TT{}.

\subsubsection{The $\MO{Instance}$ tactic}

\label{sect:instance}

When elaborating applications, recall that implicit arguments are filled in by
unification, and type class constraints by an $\MO{Instance}$ tactic which searches
the context for an instance of the required type class. 

The $\MO{Instance}$ tactic begins by trying to find a solution in the local context,
using the $\MO{Trivial}$ tactic, which attempts
to solve the current goal with each local variable in turn. It is used as a very
basic primitive for proof search, and is defined as follows:

\DM{
\AR{
\MO{Trivial}\:\mq\:
\RW{do}\:\AR{
      (\tv\Hab\ttt)\gets\MO{Context}\\
      \MO{TryAll}\:\tv
    }
\medskip\\
\MO{TryAll}\:(\vv_1,\vv_2\ldots)\:\mq\:\MO{Try}\:(\MO{Fill}\:\vv_1)\:(\MO{TryAll}\:\vv_2\ldots)
\\
\MO{TryAll}\:\langle\rangle\:\mq\:\MO{Fail}
}
}

For example, in the following function, the type class constraint required for
the $\texttt{>}$ operator is resolved by finding an instance for \texttt{Ord a}
in the local context:

\begin{SaveVerbatim}{ordlocal}

max : Ord a => a -> a -> a
max x y = if (x > y) then x else y

\end{SaveVerbatim}
\useverb{ordlocal} 

If $\MO{Trivial}$ does not find a solution, $\MO{Instance}$ begins a global search,
applying every type class instance, and attempting to resolve the instance's parameters
recursively. Given an instance $\vI$ with parameters $\ta$:

\DM{
\MO{TryInstance}\:(\vI\:\ta)\:\mq\:\RW{do}\;
\AR{
\vec{\MO{Claim}}\:(\tx\Hab\ta)\\
\ttinterp{\vI\:\tx}\\
\vec{\MO{Focus}}\:\tx;\;\vec{\MO{Instance}}
}
}

We define $\MO{Instance}$ as follows, using $\MO{AllInstances}$ to retrieve
all possible type class instances in the form required by $\MO{TryInstance}$:

\DM{
\MO{Instance}\:\mq\:
\MO{Try}\:
\AR{
\MO{Trivial}\\
(\RW{do}\:\AR{
\ti\gets\MO{AllInstances}\\
\vec{\MO{TryInstance}}\:\ti
)
}
}
}

In practice, for efficiency and to ensure that type class resolution terminates,
we constrain the search by recording which instances are defined for each type class,
and by ensuring that a recursive call of $\MO{Instance}$ is either searching for
a different class, or a structurally smaller instance of the current class. 

For example, in the following function, the constraints required by the
\tFN{show} function are resolved by a global search which locates a \tTC{Show}
instance for \tTC{List a}, which in turn requires a \tTC{Show} instance for
\texttt{a}.  In this case \texttt{a} is instantiated by \tTC{Int}, so resolution finishes
by locating a \tTC{Show} instance for \tTC{Int}:

\begin{SaveVerbatim}{showinst}

main : IO ()
main = putStrLn (show [1,2,3,4]) 

\end{SaveVerbatim}
\useverb{showinst} 

\subsection{Syntax Extensions}

We now have a complete elaborator for \IdrisM{}, covering 
dependent pattern matching definitions and expansion of implicit arguments
and type classes. In order to make the language truly general purpose, however, we
will need to add higher level extensions. Full \Idris{} is \IdrisM{} extended
with \texttt{do}-notation, idiom brackets~\cite{McBride2007}, \texttt{case} expressions,
pattern matching \texttt{let}, metavariables and tactic based theorem proving.
The majority of these extensions are straightforward transformations of high level
\Idris{} programs --- for example, \texttt{do}-notation can be reduced directly to
\IdrisM{} function applications. Some, however, require further support. In this section,
we complete the presentation of \IdrisM{} by extending it with metavariables
and $\icase$ expressions:

\DM{
\AR{
\begin{array}{rll@{\hg}rll}
\ve ::= & \ldots \\
\mid & \mvar{\vx} & (\mbox{metavariable}) &
\mid & \icase\:\ve\:\iof\:\vec{\VV{alt}} & (\mbox{case expression}) \\
\medskip\\
\VV{alt} ::= & \ve\:\fatarrow\:\ve & (\mbox{case alternative})\\
\end{array}
}
}

\subsubsection{Metavariables}

\demph{Metavariables} are terms which stand for incomplete programs. A
metavariable serves a similar
purpose to a hole binding in \TT{}, but gives rise to a global rather than a
local variable. For example, the following is an incomplete definition of
the vector append function, containing a metavariable \texttt{append\_rec}:

\begin{SaveVerbatim}{vappmv}

(++) : Vect A n -> Vect A m -> Vect A (n + m)
(++) Nil       ys = ys
(++) (x :: xs) ys = x :: ?append_rec

\end{SaveVerbatim}
\useverb{vappmv} 

This generates a new (as yet undefined)
function \texttt{append\_rec}, which takes all of the variables
in scope at the point it is used, and returns a value of the required type.
We can inspect the type of \texttt{append\_rec} at the \Idris{} prompt:

\begin{SaveVerbatim}{appendrec}

*vec> :t append_rec
append_rec : (a : Set) -> (m : Nat) -> (n : Nat) -> a -> 
             Vect a n -> Vect a m -> Vect a (n + m)

\end{SaveVerbatim}
\useverb{appendrec} 

Metavariables serve two purposes: Firstly, they aid type directed program development
by allowing a programmer direct access to the inferred types of local variables, and
the types of subexpressions. Secondly, they allow a separation of program structure
from proof details. Metavariables can be solved either by directly editing program
source, or by providing a definition elsewhere in the file. For example, we can later say:

\begin{SaveVerbatim}{appendrec_def}

append_rec a m n x xs ys = app xs ys

\end{SaveVerbatim}
\useverb{appendrec_def} 

Elaborating a metavariable involves adding a new top level definition which is applied
to all of the variables in local scope.

\DM{
\ttinterp{\mvar{\vx}}\:\mq\:
\edo{
(\tv\Hab\ttt)\gets\MO{Context}\\
\vT\gets\MO{Type}\\
\MO{TTDecl}\:(\vx\Hab(\tv\Hab\ttt)\to\vT)\\
\ttinterp{\vx\:\tv}
}
}

\subsubsection{\texttt{case} expressions}

A \texttt{case} expression allows pattern matching on intermediate values. The difficulty
in elaborating \texttt{case} expressions is that \TT{} allows matching only on
\emph{top level} values. Elaborating a \texttt{case} expression, therefore,
involves creating a new top level function standing for the expression, and applying it. 
The natural
way to implement this is to use a metavariable. For example, we have already seen
\texttt{lookup\_default}:

\useverb{listlookup}

\noindent
This elaborates as follows:

\begin{SaveVerbatim}{listlookupmv}

lookup_default : Nat -> List a -> a -> a
lookup_default i xs def = 
   let scrutinee = list_lookup i xs in ?lookup_default_case

lookup_default_case i xs def Nothing  = def
lookup_default_case i xs def (Just x) = x

\end{SaveVerbatim}
\useverb{listlookupmv} 

To elaborate a $\icase$ expression involves \texttt{let} binding the scrutinee of
the $\icase$ expression and creating a metavariable $\VV{xcase}$ to implement the
pattern matching. Then the elaborator builds a new pattern matching definition for $\VV{xcase}$
and elaborate it:

\DM{
\AR{
\ttinterp{\icase\:\ve\:\iof\:\vec{\VV{alt}}}\:\mq\:
\edo{
\ttinterp{\ilet{\VV{scrutinee}}{\ve}\:\mvar{\VV{xcase}}}\\
(\tv\Hab\ttt)\gets\MO{Context}\\
\vec{\MO{MkCase}}\:\tv\:\VV{alt}\\
}
\medskip\\
\MO{MkCase}\:\tv\:(\vl\:\fatarrow\:\vr)\:\mq\:\MO{Elab}\:(\VV{xcase}\:\tv\:\vl\:=\:\vr)
}
}

Any nested $\icase$ expressions will be handled by the recursive call to
$\MO{Elab}$. Again, because of the way we have set up an Embedded Domain Specific
Language for describing elaboration, we have been able to implement a new higher
level language feature in terms of elaboration of lower level language features.


%\subsubsection{Tactic-implicit arguments}


%\subsubsection{Pairs and Dependent Pairs}

% Other extensions: do notation, idiom brackets, pairs, etc, are easily expressed
% by transformations of the high level syntax



\section{Reversing Elaboration}

\label{sect:delab}

As well as translating from \Idris{} to \TT{}, so that programs can be type
checked and evaluated, it is valuable to define the reverse transformation. This
serves two principal purposes:

\begin{itemize}
\item To assist the user, it is preferable that the results of evaluation, and any
error messages produced by the elaborator, are presented in \Idris{} syntax
rather than \TT{}.
\item For correctness, we would like to ensure as far as possible that the 
result of elaboration is equivalent to the original program. Informally, we can
achieve this by checking that reversing the elaboration process yields the original
program (with implicit arguments expanded).
\end{itemize}

\noindent
In this section, we describe the process for reversing elaboration and the required
properties of the elaboration process as a whole. Fortunately, translating from
\TT{} to \Idris{} is significantly easier than \Idris{} to \TT{}, because it is
primarily \emph{erasing} information.

\subsection{From \TT{} to \Idris{}}

We define a meta-operation $\uninterp{\vt}$, which converts a \TT{} expression
$\vt$ to an \Idris{} expression which would elaborate to $\vt$:

\DM{
\AR{
\begin{array}{rl}
\uninterp{\vx} &\mq\:\vx\\
\uninterp{\vc} &\mq\:\vc\\
\uninterp{\vx\;\ta} &\mq\:\vx\;(\vec{\MO{Impl}}\:\vx\:\ta) \\
\uninterp{\vf\:\va} &\mq\:\uninterp{\vf}\:\uninterp{\va}\\
\uninterp{\lam{\vx}{\vT}\SC\ve} &\mq \:\ilam{\vx}\uninterp{\ve}  \\
\uninterp{\all{\vx}{\vT}\SC\ve} &\mq \:\piexp{\vx}{\uninterp{\vT}}\uninterp{\ve}  \\
\uninterp{\LET\:\vx\defq\vt\Hab\vT\SC\ve} &\mq  
\ilet{\vx}{\uninterp{\vt}}\uninterp{\ve}\\
\end{array}
\medskip\\
\begin{array}{rll}
\MO{Impl}\:\vx\:\va_i\:\mq
&
\iarg{\vn}{\ttinterp{\va_i}}&
\mbox{(if the $i$th argument to $\vx$ is implicit argument $\vn$)}\\
& \carg{\ttinterp{\va_i}} &
\mbox{(if the $i$th argument to $\vx$ is a constraint argument)}\\
& \ttinterp{\va_i} & \mbox{(otherwise)}
\end{array}
}
}

\noindent
This is mostly a straightforward
traversal of the \TT{} expression, translating directly to an \Idris{} equivalent.
The interesting case is for applications of named functions,
$\uninterp{\vx\:\ta}$, where the arguments are translated to either implicit,
constraint or explicit arguments according to the definition of $\vx$.
Since only type declarations are allowed to have implicit or constraint arguments,
and $\uninterp{\cdot}$ translates \emph{expressions},
all function types are assumed to take explicit arguments.

We also define an operation $\MO{Unelab}$, which translates \TT{} declarations
to corresponding \Idris{} declarations. This generates data declarations and
pattern matching definitions only --- it makes no attempt to reconstruct 
\texttt{class}
or \texttt{instance} declarations, or rebuild \texttt{case} expressions.
First, we define the reverse elaboration of type declarations, which must reconstruct
which arguments are implicit or constraint arguments:

\DM{
\AR{
\begin{array}{l}
\MO{UnelabType}\:(\vx\Hab\vt)\:\mq\:\vx\Hab\MO{UnelabTyDecl}\:0\:\vt
\end{array}
\medskip\\
\begin{array}{ll}
\MO{UnelabTyDecl}\:\vi\:(\all{\vx}{\vT}\SC\ve)
\mq & \piimp{\vx}{\uninterp{\vT}}(\MO{UnelabTyDecl}\:(\vi+1)\:\ve) \\
 & \hg\mbox{(if the $i$th argument to $\vx$ is an implicit argument)} \\
 & \piconst{\uninterp{\vT}}(\MO{UnelabTyDecl}\:(\vi+1)\:\ve) \\
 & \hg\mbox{(if the $i$th argument to $\vx$ is a constraint argument)} \\
 & \piexp{\vx}{\uninterp{\vT}}(\MO{UnelabTyDecl}\:(\vi+1)\:\ve) \\
 & \hg\mbox{(otherwise)} \\
\end{array}
}
}

\noindent
Using this, we define $\MO{Unelab}$ for top level declarations. For pattern
matching clauses reversing elaboration proceeds as follows, discarding the explicit
pattern variable bindings and applying $\MO{UnelabType}$ to reconstruct
the type declaration:

\DM{
\AR{
\MO{Unelab}\:(\vx\Hab\vt)\:\mq\:\MO{UnelabType}\:(\vx\Hab\vt)\\
\MO{Unelab}\:(\pat{\tx}{\tU}\SC\FN{f}\:\tts\:=\:\ve)
\:\mq\:\uninterp{\FN{f}\:\tts}\:=\:\uninterp{\ve}
}
}

\noindent
For data type declarations, reverse elaboration proceeds as follows, applying $\MO{UnelabType}$
for each of the top level type declarations:

\DM{
\AR{
\MO{Unelab}\:(\Data\;\TC{T}\:(\tx\Hab\ttt)\Hab\vT\;\Where\;\vec{\VV{cons}})
\:\\
\hg\hg\mq\:\idata\:\MO{UnelabType}\:(\TC{T}\Hab\all{\tx}{\ttt}\SC\vT)\:\iwhere\:
(\vec{\MO{UnelabType}}\:\vec{\VV{cons}})
}
}


\subsection{Elaboration Properties}

Elaboration satisfies two important properties. We limit our discussion of
these properties to \IdrisM{} without type classes, i.e. elaboration of type 
declarations, functions, and data types. This is primarily because there is not
enough information in a \TT{} program alone to reverse elaboration fully. However, since
elaboration of the higher level \Idris{} constructs is implemented in terms of
the lower level \IdrisM{} constructs, we should not consider this a serious
limitation.

Informally stated, the properties that elaboration satisfies are
i) that if elaboration is successful, the
resulting program is a well-typed \TT{} program; ii) elaboration preserves the
meaning of the original \Idris{} program. The first property is true by the
definition of elaboration --- elaboration fails if it attempts to construct an ill-typed
term at any point. Furthermore, the development calculus \TTdev{} ensures that 
partial constructions are well-typed.
The second property can be stated as the following conjecture:
\\

\noindent
\textbf{Conjecture: Preservation of meaning}\\
Given an \IdrisM{} declaration $\vd$, which is either a type declaration, a
pattern match clause, or a data type declaration,
if $\MO{Elab}\:\vd\:\mq\:\vt$, and $\MO{Unelab}\:\vt\:\mq\:\vd'$,
then $\MO{Match}\:\vd\:\vd'$ produces a valid match.
\\

The output of $\MO{Unelab}$ is not necessarily equal to the input of $\MO{Elab}$
because elaboration fills in placeholder subexpressions. Therefore it suffices for
the input to match the output.

We have not yet proved this conjecture. Its truth depends on the implementation
of $\MO{Elab}$ faithfully translating each construct, and we observe that the
present description of $\MO{Elab}$ elaborates each non-placeholder subexpression
according to the structure of the expression. However, since the truth of this
conjecture is crucial to the correctness of the implementation, the elaborator
checks \emph{dynamically} that meaning is preserved by evaluating
$\MO{Unelab}\:\vt$ and matching the input against the result. In practice, this
does not have a significant impact on performance.



%\section{Compiling}



%\section{Syntax Extensions}

\Idris{} supports the implementation of Embedded Domain Specific Languages (EDSLs) in
several ways~\cite{res-dsl-padl12}. One way, as we have already seen, is through
extending \texttt{do} notation. Another important way is to allow extension of the core
syntax. In this section we describe two ways of extending the syntax: \texttt{syntax}
rules and \texttt{dsl} notation.

\subsection{\texttt{syntax} rules}

We have seen \texttt{if...then...else} expressions, but these
are not built in --- instead, we define a function in the prelude\ldots

\begin{SaveVerbatim}{boolelim}

boolElim : (x:Bool) -> |(t : a) -> |(f : a) -> a; 
boolElim True  t e = t;
boolElim False t e = e;

\end{SaveVerbatim}
\useverb{boolelim}

\noindent
\ldots and extend the core syntax with a \texttt{syntax} declaration:

\begin{SaveVerbatim}{syntaxif}

syntax if [test] then [t] else [e] = boolElim test t e;

\end{SaveVerbatim}
\useverb{syntaxif}

\noindent
The left hand side of a \texttt{syntax} declaration describes the syntax rule, and the right
hand side describes its expansion. The syntax rule itself consists of:

\begin{itemize}
\item \textbf{Keywords} --- here, \texttt{if}, \texttt{then} and \texttt{else}, which must
be valid identifiers
\item \textbf{Non-terminals} --- included in square brackets, \texttt{[test]}, \texttt{[t]}
and \texttt{[e]} here, which stand for arbitrary expressions. To avoid parsing ambiguities, 
these expressions cannot use syntax extensions at the top level (though they can be used
in parentheses).
\item \textbf{Names} --- included in braces, which stand for names which may be bound
on the right hand side.
\item \textbf{Symbols} --- included in quotations marks, e.g. \texttt{":="}. This can
also be used to include reserved words in syntax rules, such as \texttt{"let"} or \texttt{"in"}.
\end{itemize}

\noindent
The limitations on the form of a syntax rule are that it must include at least one
symbol or keyword, and there must be no repeated variables standing for non-terminals.
Rules can use previously defined rules, but may not be recursive.
The following syntax extensions would therefore be valid:

\begin{SaveVerbatim}{syntaxex}

syntax [var] ":=" [val]              = Assign var val;
syntax [test] "?" [t] ":" [e]        = if test then t else e;
syntax select [x] from [t] where [w] = SelectWhere x t w;
syntax select [x] from [t]           = Select x t;

\end{SaveVerbatim}
\useverb{syntaxex}

\noindent
Syntax macros can be further restricted to apply only in patterns (i.e., only on the left
hand side of a pattern match clause) or only in terms (i.e. everywhere but the left hand side
of a pattern match clause) by being marked as \texttt{pattern} or \texttt{term} syntax
rules. For example, we might define an interval as follows, with a static check
that the lower bound is below the upper bound using \texttt{so}:

\begin{SaveVerbatim}{interval}

data Interval : Type where
   MkInterval : (lower : Float) -> (upper : Float) -> 
                so (lower < upper) -> Interval

\end{SaveVerbatim}
\useverb{interval}

\noindent
We can define a syntax which, in patterns, always matches \texttt{oh} for the proof 
argument, and in terms requires a proof term to be provided:

\begin{SaveVerbatim}{intervalsyn}

pattern syntax "[" [x] "..." [y] "]" = MkInterval x y oh
term    syntax "[" [x] "..." [y] "]" = MkInterval x y ?bounds_lemma

\end{SaveVerbatim}
\useverb{intervalsyn} 

\noindent
In terms, the syntax \texttt{[x...y]} will generate a proof obligation
\texttt{bounds\_lemma} (possibly renamed).

Finally, syntax rules may be used to introduce alternative binding forms. For
exampe, a \texttt{for} loop binds a variable on each iteration:

\begin{SaveVerbatim}{forloop}

syntax for {x} "in" [xs] [body] = forLoop xs (\x => body)
  
main : IO ()
main = do for x in [1..10] do
              putStrLn ("Number " ++ show x)
          putStrLn "Done!"

\end{SaveVerbatim}
\useverb{forloop} 

\noindent
Note that we have used the \texttt{\{x\}} form to state that \texttt{x} represents
a bound variable, substituted on the right hand side. We have also put \texttt{"in"} in
quotation marks since it is already a reserved word.

\input{dsl}



\section{Related Work}

\label{sect:related}

Dependently typed programming languages have becoming more prominent in recent
years as tools for verifying software correctness, and several experimental
languages are being developed, in particular Agda \cite{norell2007thesis}, Epigram
\cite{McBride2004a,Levitation2010} and Trellys \cite{Kimmell2012}.

An earlier implementation of \Idris{} was built on the \Ivor{} proof engine 
\cite{Brady2006b}. This implementation differed in one important way --- unlike
the present implementation, there was limited separation between the type theory
and the high level language. The type theory itself supported implicit syntax and
unification, with high level constructs such as the \texttt{with} rule implemented
directly. Two important disadvantages were found with this approach, however: 
firstly, the type checker is much more complicated when combined with
unification, making it harder to maintain; secondly, adding new high level features
requires the type checker to support those features directly. In contrast, elaboration
by tactics gives a clean separation between the low level and high level languages.

The Agda implementation is based on a type theory with
implicit syntax and pattern matching --- Norell gives an algorithm for type checking
a dependently typed language with pattern matching and metavariables 
\cite{norell2007thesis}. Unlike the present \Idris{} implementation, metavariables
and implicit arguments are part of the type theory. This has the advantage that
implicit arguments can be used more freely, at the expense of complicating the type
system.

Epigram \cite{McBride2004a} and Oleg \cite{McBride1999} 
have provided much of the inspiration for the \Idris{} elaborator. Indeed,
the hole and guess bindings of \TTdev{} are taken directly from Oleg.
\Epigram{} does not implement pattern matching directly, but rather translates
pattern matching into elimination rules \cite{McBride2002}. This has the
advantage that
elimination rules provide termination and coverage proofs \emph{by construction}.
Furthermore they simplify implementation of the evaluator and provide easy
optimisation opportunities \cite{Brady2003}. However, it requires the
implementation of extra machinery for constructor manipulation
\cite{McBride2006} and so we have avoided it in the present implementation.

%Comparison with GHC's type system and type checker. \cite{Vytiniotis2011}

%[Observation: separate elaboration and type checking, sort of like in GHC which
%type checks the high level language and produces a type correct core language.
%Elaboration is effectively a type checker for the high level language, so we have
%a hope of providing reasonable error messages related to the original code.]

\section{Conclusion}

\label{sect:conclusion}

In this paper, I have given an overview of the programming language \Idris{},
and its core type theory \TT{}, giving a detailed algorithm for translating
high level programs into \TT{}.
\TT{} itself is deliberately small and simple, and the design has deliberately
resisted innovation so that we can rely on existing metatheoretic properties
being preserved. The kernel of the \Idris{} implementation consists of a type checker
and evaluator for \TT{} along with a pattern match compiler, which are implemented
in under 1000 lines of Haskell code. It is important that this kernel remains small
--- the correctness of the language implementation relies to a large extent on
the correctness of the underlying type system, and keeping the implementation small
reduces the possibility of errors.

The approach we have taken to implementing the high level language, implementing
a tactic language as an embedded DSL, allows us to
build programs on top of a small and unchanging kernel, rather than extending 
the core language to deal with implicit syntax, unification and type classes.
High level \Idris{} features are implemented by describing the corresponding
sequence of tactics to build an equivalent program in \TT{}, via a development
calculus of incomplete terms, \TTdev{}. A significant advantage we have found with
this approach is that higher level features can easily be implemented in terms
of existing elaborator components. For example, once we have implemented elaboration
for data types and functions, it is easy to add several features:

\begin{itemize}
\item \textbf{Type classes}: A dictionary is merely a record containing the
functions which implement a type class instance. Since we have a tactic based
refinement engine, we can implement type class resolution as a tactic.
\item \textbf{\texttt{where} clauses}: We have access to local variables and
their types, so we can
elaborate \texttt{where} clauses at the point of definition simply by lifting
them to the top level. 
\item \textbf{\texttt{case} expressions}: Similar to \texttt{where} clauses,
these are implemented by lifting the branches out to a top level function.
\end{itemize}

We do not need to make any changes to the core language type system in order to 
implement these high level features. 
Other high level features such as dependent records, tuples and monad comprehensions
can be added equally easily --- and indeed have been added in the full implementation.  
Furthermore, 
since we have used a tactic-based EDSL to elaborate \Idris{} to \TT{}, it is
possible to expose the tactic language to the programmer. This opens up the
possibility of implementing domain specific decision procedures, or implementing
user defined tactics in a style similar to Coq's \texttt{Ltac} language \cite{Delahaye2000}.
Although \TT{} is primarily intended as a core language for \Idris{}, 
its rich type system also allows it to capture a number of high level languages, especially
when augmented with primitive operators. 

We have not discussed the performance of the elaboration
algorithm, or described how \Idris{} compiles to executable code. In practice,
we have found performance to be acceptable --- for example, the \Idris{}
library (25 files, 3532 lines of code in total at the time of writing)
elaborates in around 10 seconds\footnote{On a MacBook Pro, 2.8GHz Intel Core 2
Duo, 4Gb RAM}. Profiling suggests that the main bottleneck is locating holes
in a proof term, which can be improved by choosing a better representation
for proof terms, perhaps based on a zipper \cite{Huet1997}. Compilation is made
straightforward by the Epic library \cite{brady2011epic}, with I/O and foreign
functions handled using command-response interaction trees \cite{Hancock2000}.

The objective of this implementation of \Idris{} is to provide a platform
for experimenting with realistic, general purpose programming with dependent
types, by implementing a Haskell-like language augmented with \emph{full}
dependent types. 
In this paper, we have seen how such a high level language can be implemented
by building on top of a small, well-understood, easy to reason about type
theory. 
However, a programming language implementation is not an end in itself. 
Programming languages exist to support research and practice in many different
domains. In future work, therefore, I plan to apply domain specific
language based techniques to realistic problems
in important safety critical domains such as security and network protocol
design and implementation. In order to be successful, this will require a language which
is expressive enough to describe protocol specifications at a high level, and robust 
enough to guarantee correct implementation of those protocols. \Idris{}, I believe,
is the right tool for this work.



%\subsection{Further Work}

%[Would the EDSL approach work in other languages? Adding components of DTP
%to imperative languages, say, using \TT{} as a verified core.
%\Idris{} implementation as the beginning of a project to explore practical
%DTP --- systems, protocols, security especially. And just having full
%dependent types for lightweight correctness guarantees.]


\section*{Acknowledgements}

This work was funded by the Scottish Informatics and Computer Science
Alliance (SICSA) and by EU Framework 7 Project No. 248828 (ADVANCE).
My thanks to Kevin Hammond and Vilhelm Sj\"{o}berg for their comments 
on an earlier draft of this paper.

\bibliographystyle{jfp}
\bibliography{library.bib}

\appendix

%\section{TODOs}
%\listoftodos{}

%\section{Elaboration meta-operations}

%It's possible that it would be useful to have a quick reference of meta-operations
%used by the elaborator here.

%They are: $\ttinterp{\cdot}$, $\MO{Elab}$, $\MO{TTDecl}$,
%$\MO{NewProof}$, $\MO{NewTerm}$,
%$\MO{Term}$, $\MO{Type}$, $\MO{Context}$, $\MO{Patterns}$, $\MO{Lift}$, $\MO{Expand}$.

%\input{code}

\end{document}
